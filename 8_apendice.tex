\chapter{Instruções para execução do fluxo de trabalho}
\label{ch:apendice}

A ferramenta desenvolvida ao longo do presente projeto pode ser encontrada no link \href{https://concilia.ufsc.br/}{https://concilia.ufsc.br/}

% [TODO 2: Descrever a estrutura e diretórios do repositório]
O código para a reprodução do fluxo proposto pode ser encontrado no link $\href{https://github.com/ConcilIA-EGOV/predictor_model_in_law_judgments}{https://github.com/ConcilIA-EGOV/predictor_model_in_law_judgments}$.

% [TODO 2: Descrever a melhor como usar o(s) repositório(s)]
Para a correta execução, é necessário ter a versão 3.10 do \textit{Python}, E o gerenciador de dependências \textit{pip}.
Após instalado, crie um ambiente virtual e instale as dependências necessárias:


\begin{lstlisting}
  -  python -m venv caminho/para/novo/ambiente/virtual
  -  source caminho/para/novo/ambiente/virtual/bin/activate
  -  pip install -r requirements.txt
\end{lstlisting}

Com o ambiente virtual aberto e devidamente configurado, basta digitar o comando \textit{make} para ser informado das opções de uso.

------------------------------------------------------------------------------


% [TODO 2: Descrever a estrutura e uso do site]

% [? TODO: Entender melhor os apêndices]

\begin{table}[!ht]
    \centering
    \begin{tabular}{|| m{3 cm} | m{2 cm} | m{9.5 cm} ||}
        \hline
        Biblioteca & Versão  & Usabilidade Principal \\ \hline
        \hline
        \hline
        cloudpickle & 3.1.1 & Serialização de objetos Python \\ \hline
        contourpy & 1.3.2 & Geração de contornos para visualização \\ \hline
        cycler & 0.12.1 & Criação de ciclos para estilos de plotagem \\ \hline
        fonttools & 4.57.0 & Manipulação de fontes tipográficas \\ \hline
        graphviz & 0.20.3 & Visualização de grafos e árvores \\ \hline
        imbalanced-learn & 0.13.0 & Técnicas de balanceamento de dados \\ \hline
        imblearn & 0.0 & Interface para imbalanced-learn \\ \hline
        joblib & 1.5.0 & Persistência e paralelização de objetos Python \\ \hline
        kiwisolver & 1.4.8 & Resolução de restrições para gráficos \\ \hline
        llvmlite & 0.44.0 & Suporte à compilação JIT (usado pelo numba) \\ \hline
        matplotlib & 3.10.1 & Visualização de dados \\ \hline
        numba & 0.61.2 & Compilação JIT para aceleração de código Python \\ \hline
        numpy & 2.2.5 & Operações numéricas e manipulação de arrays \\ \hline
        packaging & 25.0 & Utilitários para empacotamento de pacotes Python \\ \hline
        pandas & 2.2.3 & Manipulação e análise de dados tabulares \\ \hline
        pillow & 11.2.1 & Processamento de imagens \\ \hline
        pip & 25.1.1 & Gerenciador de pacotes Python \\ \hline
        pyparsing & 3.2.3 & Análise sintática de expressões \\ \hline
        python-dateutil & 2.9.0 & Manipulação avançada de datas \\ \hline
        pytz & 2025.2 & Suporte a fusos horários \\ \hline
        scikit-learn & 1.6.1 & Modelagem preditiva e algoritmos de machine
        learning \\ \hline
        scipy & 1.15.2 & Computação científica e estatística \\ \hline
        shap & 0.47.2 & Explicabilidade de modelos de machine learning \\ \hline
        six & 1.17.0 & Compatibilidade entre Python 2 e 3 \\ \hline
        sklearn-compat & 0.1.3 & Compatibilidade entre versões do
        scikit-learn \\ \hline
        slicer & 0.0.8 & Utilitário para slicing de dados \\ \hline
        threadpoolctl & 3.6.0 & Controle de pools de threads \\ \hline
        tqdm & 4.67.1 & Barra de progresso para loops \\ \hline
        typing\_extensions & 4.13.2 & Extensões de tipagem para Python \\ \hline
        tzdata & 2025.2 & Base de dados de fusos horários \\ \hline
    \end{tabular}
    \caption{Detalhamento das bibliotecas python usadas}
    \fonte{Elaborado pelo autor}
    \label{tab:python_versions}
\end{table}