\chapter{Introdução}

O Poder Judiciário brasileiro enfrenta um cenário de crescente litigiosidade, conforme aponta o relatório "Justiça em Números" do Conselho Nacional de Justiça  \cite{da2024relatorio}.
Em resposta a este desafio, o CNJ tem promovido duas agendas estratégicas principais: a primeira, consolidada pela Resolução nº 125/2010, foca na promoção de meios autocompositivos de solução de conflitos, como a conciliação e a mediação \cite{RCNJ1252010}.

A segunda, impulsionada por resoluções como a de nº 332/2020 \cite{RCNJ3322020}, prioriza o uso de tecnologias e Inteligência Artificial (IA) para modernizar e otimizar os serviços judiciais, bem como a Resolução nº 615 de 2025, que estabelece diretrizes para o desenvolvimento, utilização e governança de soluções desenvolvidas com recursos de inteligência artificial no Poder Judiciário \cite{RCNJ6152025}.

Embora eficazes, as sessões de conciliação frequentemente encontram um obstáculo prático: a assimetria de expectativas entre as partes quanto ao valor da indenização.
Autores de processos por vezes possuem uma percepção superestimada do valor que lhes é devido, enquanto os réus tendem a oferecer quantias consideravelmente inferiores.
Essa disparidade dificulta o consenso e, consequentemente, a celebração de acordos, prolongando o litígio.

Diante deste cenário, define-se o problema de pesquisa desta dissertação através da seguinte questão:

\textbf{É possível desenvolver um modelo computacional capaz de predizer valores de indenização por dano moral com precisão e transparência suficientes para auxiliar na tomada de decisão em sessões de conciliação?}

A hipótese central deste trabalho é que a utilização de algoritmos de Aprendizado de Máquina Supervisionado, especificamente modelos de regressão treinados em dados históricos estruturados, permite estimar o valor provável de condenação.

Adicionalmente, assume-se que a integração de técnicas de Inteligência Artificial Explicável (XAI) é fundamental para mitigar o efeito "caixa-preta" dos modelos, fornecendo às partes a justificativa necessária para aceitar a sugestão do sistema.

O projeto Concil-IA \cite{sabo2025avancos}, desenvolvido na Universidade Federal de Santa Catarina (UFSC) em parceria com o Juizado Especial Cível (JEC), serve como base para a verificação desta hipótese.
Utilizando uma base de dados de 1.174 sentenças judiciais sobre conflitos aéreos, ferramentas de Aprendizado de Máquina (\textit{Machine Learning}),
como a biblioteca \textit{Scikit-learn}, para treinar modelos de regressão,
e o método \textit{SHapley Additive exPlanations} (SHAP) para interpretar as predições geradas \cite{salih2025perspective};
este estudo investiga a viabilidade técnica dessa abordagem preditiva.

Ressalta-se, contudo, que a solução é proposta como um instrumento de auxílio à tomada de decisão, e não como um substituto da análise e deliberação humana, respeitando-se o devido processo legal e a autonomia dos envolvidos.

\label{ch:intro}
\section{Objetivos}
\subsection{Objetivo geral}
\label{sub:objetivo_geral}
Desenvolver e implementar um modelo de Inteligência Artificial Explicável capaz de prever o valor de danos morais decididos em julgamentos relacionados a direitos do consumidor em voos comerciais, contribuindo para a explicabilidade de decisões judiciais e modelos de IA, e permitindo disponibilizar essa funcionalidade ao público por meio de uma interface web.
\subsection{Objetivos específicos}
\label{sub:objetivos_spec}
\begin{enumerate}[label=(\roman*)]
    \item Desenvolver e treinar um modelo de IA utilizando dados estruturados obtidos de sentenças jurídicas para prever valores de danos morais.
    \item Analisar diferentes algoritmos e abordagens para construção do modelo, avaliando métricas de desempenho e identificando razões para diferenças entre eles.
    \item Integrar o modelo a uma plataforma web já existente, permitindo que usuários consultem previsões de valores de danos morais com base em dados fornecidos.
    \item Compilar e documentar todo o processo, desde a obtenção e estruturação dos dados até a análise dos resultados e as etapas de integração ao site.
    \item Analisar a importância das \textit{features} do melhor modelo encontrado.
\end{enumerate}


A metodologia adotada para atingir tais objetivos combina
a estruturação de um \textit{dataset} proprietário,
a avaliação comparativa de algoritmos de regressão da biblioteca \textit{Scikit-learn}
e a implementação de uma interface web explicável. 
A validação dos resultados considera não apenas métricas de erro estatístico, mas também a interpretabilidade das \textit{features} pelo método SHAP.

\section{Organização do trabalho}
Os próximos capítulos deste trabalho estão organizados da seguinte forma:
\begin{enumerate}
    \item O Capítulo 2 consiste na fundamentação teórica, onde serão abordados conceitos fundamentais para a compreensão do trabalho, tais como inteligência artificial, explicabilidade, aprendizado de máquina e os diferentes modelos regressores que serão utilizados.
    \item O Capítulo 3 aborda os trabalhos relacionados, comparando a abordagem desta pesquisa com a de outros pesquisadores em temas correlatos.
    \item O Capítulo 4 explica a metodologia empregada para produzir, melhorar e avaliar os resultados.
    \item O Capítulo 5 expõe e discute os resultados obtidos até então.
    \item Por fim, o Capítulo 6 apresenta as conclusões retiradas deste trabalho, como também apontar os próximos passos do projeto.
\end{enumerate}

Na elaboração dessa dissertação, modelos de linguagem como OpenAI Chat GPT, Gemini AI e Manus, foram utilizados para revisão e estruturação do texto.