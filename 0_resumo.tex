  % Aqui deve ser inserido um resumo de 150 a 500 palavras (limitação de tamanho dada pela BU).
  % A linguagem deve ser português e a hifenização já foi alterada.
  % O resumo em português deve preceder o resumo em inglês, mesmo que o trabalho seja escrito em inglês.
  % A BU também diz que deve ser usada a voz ativa e o discurso deve ser na 3ª pessoa.
  % A estrutura do resumo pode ser similar a estrutura usada em artigos: Contexto -- Problema -- Estado da arte -- Solução proposta  -- Resultados.
\begin{resumo}[Resumo]
O judiciário brasileiro enfrenta um crescente volume de litigiosidade, impulsionando a adoção de tecnologias de Inteligência Artificial (IA) para otimizar a resolução de conflitos.

Neste contexto, o presente trabalho, derivado das atividades desenvolvidas no projeto Concil-IA, teve como objetivo central o desenvolvimento  de um modelo preditivo de regressão para estimar valores de indenização por danos morais em disputas consumeristas de transporte aéreo, com foco em sua explicabilidade.

A metodologia envolveu o pré-processamento de uma base de dados com 1.174 sentenças judiciais, o treinamento e a avaliação de algoritmos de aprendizado de máquina, como o \textit{DecisionTreeRegressor}, e a aplicação da técnica de Inteligência Artificial Explicável (XAI) \textit{SHapley Additive exPlanations} (SHAP) para interpretar os resultados.

Como resultado, foi desenvolvido um modelo funcional com um Erro Quadrático Médio Raiz (RMSE) de R\$ 2.260,57 e um Erro Absoluto Médio (MAE) de R\$ 1.577,21, capaz de gerar predições de valor e identificar os fatores de maior impacto em cada decisão.
Conclui-se que a abordagem é tecnicamente viável e oferece um instrumento de auxílio para as partes em processos de conciliação, fornecendo estimativas transparentes e justificáveis que contribuem para a promoção da pacificação social e para o avanço dos sistemas de resolução de disputas online (ODR).
  \vspace{\baselineskip} 
  % Atenção! a BU exige separação através de ponto (.). Ela recomanda de 3 a 5 keywords
  \textbf{Palavras-chave:} Inteligência Artificial. Aprendizado de Máquina. Regressão. Explicabilidade. Resolução de Disputas Online.
\end{resumo}

\begin{abstract}
The Brazilian judiciary faces a increasing volume of litigation, driving the adoption of Artificial Intelligence (AI) technologies to optimize dispute resolution.

In this context, this work, derived from the activities developed in the Concil-IA project, had as its central objective the development of a predictive regression model to estimate compensation amounts for moral damages in air transport consumer disputes, with a focus on their explainability.
The methodology involved the preprocessing of a database with 1,174 court judgments, the training and evaluation of machine learning algorithms, such as \textit{DecisionTreeRegressor}, and the application of the Explainable Artificial Intelligence (XAI) technique \textit{SHapley Additive exPlanations} (SHAP) to interpret the output.
As a result, a functional model was developed with a Root Mean Square Error (RMSE) of R\$ 2,260,57 and a Mean Absolute Error (MAE) of R\$ 1,577,21 Reais, capable of generating value predictions and identifying the factors with the greatest impact on each decision.

The conclusion is that the approach is technically feasible and offers a tool to assist parties in conciliation proceedings, providing transparent and justifiable estimates that contribute to the promotion of social pacification and the advancement of online dispute resolution (ODR) systems.

  \vspace{\baselineskip} 
  \textbf{Keywords:} Artificial Intelligence. Machine Learning. Regression. Explainability. Online Dispute Resolution.
\end{abstract}