\chapter{Metodologia}
\label{ch:metodologia}
O desenvolvimento deste trabalho seguiu uma abordagem quantitativa para o modelo e qualitativa para a explicabilidade; com a aplicação de técnicas de Ciência de Dados e Aprendizado de Máquina para a construção de um modelo preditivo de regressão e técnicas de Inteligência Artificial Explicável para permitir a explicação e interpretação de dados por usuários leigos.

O processo metodológico do Projeto Concil-IA foi estruturado em um \textit{pipeline} de múltiplas etapas:

\begin{figure}[ht]
    \centering
    \includegraphics[width=0.5\linewidth]{images/Concil-IA-Pipeline.png}
    \caption{Pipeline Concil-IA}
    \fonte{Elaborado pelo Autor}
    \label{fig:concilia_pipeline}
\end{figure}

Já este trabalho contempla apenas as etapas:
\begin{figure}[ht]
    \centering
    \includegraphics[width=0.5\linewidth]{images/Project-Pipeline.png}
    \caption{Pipeline desse projeto}
    \fonte{Elaborado pelo Autor}
    \label{fig:project-pipeline}
\end{figure}

As seções a seguir detalham cada um dos passos deste trabalho.

% [TODO: Fazer uma pequena introdução de cada etapa]

\section{Base de Dados e Ferramentas Computacionais}

\subsection{Base de Dados}
\label{ch:dataset}

A base de dados corresponde a 1.174 sentenças provenientes do Juizado Especial Cível lotado na UFSC que versam sobre falhas no serviço de transporte aéreo (Direito do Consumidor).
Essas sentenças foram coletadas manualmente por outros membros do projeto Concil-IA, na frente A.
Após o processo de anonimização, os membros da equipe pertencentes à frente B extraíram variáveis (ou fatores) e as estruturaram com base nas descrições encontradas no Quadro \ref{tab:feature_desc}

\begin{table}[ht]
    \centering
    \begin{tabular}{|| m{6 cm} | m{10 cm} ||}
        \hline
        Feature & Descrição \\
        \hline
        \hline
        direito de arrependimento & O consumidor tentou cancelar ou alterar a
        compra no prazo de 7 dias, mas não lhe foi permitido fazê-lo. \\ \hline
        descumprimento de oferta & A empresa fornecedora não cumpriu com o
        ofertado ao consumidor(a), seja no valor da passagem, seja no assento no
        avião ou situações correlatas. \\ \hline
        extravio definitivo & Uma ou mais bagagens foram extraviadas e nunca
        foram recuperadas. \\ \hline
        extravio temporário & Uma ou mais bagagens foram extraviadas e
        posteriormente recuperadas. \\ \hline
        intervalo extravio temporário & O tempo decorrido até recuperar a
        bagagem, caso haja extravio temporário \\ \hline
        violação, furto ou avaria & Adulteração da bagagem ou um item dela \\ \hline
        cancelamento/alteração de destino & O consumidor não foi levado ao
        destino inicial \\ \hline
        atraso de vôo & O consumidor foi levado ao destino inicial, mas com
        atraso maior que 4 horas \\ \hline
        intervalo atraso & O tempo decorrido até o consumidor chegar ao seu
        destino, caso haja atraso de vôo \\ \hline
        culpa exclusiva consumidor & Todos os problemas relatados decorrem de
        ações ou inações do consumidor. \\ \hline
        condições climáticas / fechamento & Evento imprevisível e inevitável que
        comprovadamente impediu o aeroporto de operar. \\ \hline
        no show & Cancelamento automático do vôo de retorno em razão exclusiva
        do não comparecimento à viagem de ida, sem concordância do
        consumidor. \\ \hline
        overbooking & Venda de assentos em quantidade superior a comportada pela
        aeronave fazendo com que o consumidor não embarcasse. \\ \hline
        assistência cia aérea & A companhia aérea forneceu auxílio concreto ao
        consumidor para enfrentar o problema de atraso de voo ou cancelamento
        sem realocação/alteração de destino, e essa assistência inclui
        hospedagem, alimentação, ou transporte alternativo. \\ \hline
        hipervulnerável & O consumidor era idoso, gestante, possuía deficiência,
        fazia uso de medicamentos ou estava acompanhado de criança (entende-se
        criança a(o) infante que tem até 12 anos incompletos). \\ \hline
    \end{tabular}
    \caption{Descrição das Features}
    \fonte{Elaborado pelo autor}
    \label{tab:feature_desc}
\end{table}

Os dados, anonimizados em conformidade com a Lei Geral de Proteção de Dados (LGPD), foram estruturados e extraídos para arquivos no formato CSV (\textit{comma-separated values}).
O processo de anonimização garantiu que informações sensíveis, como nomes das partes, fossem removidas, atendendo às exigências éticas para pesquisa científica.	
O Quadro \ref{tab:features_distribution} mostra a distribuição de valores para cada feature encontrada na base de dados extraída.

\begin{table}[!ht]
    \centering
    \begin{tabular}{|| m{6 cm} | m{2 cm} | m{2 cm} ||}
        \hline
        Features: & N & S \\ \hline
        \hline
        \hline
        direito de arrependimento & 98,65\% & 1,35\% \\ \hline
        descumprimento de oferta & 99,15\% & 0,85\% \\ \hline
        extravio definitivo & 96,86\% & 3,14\% \\ \hline
        extravio temporário & 86,84\% & 13,16\% \\ \hline
        violação furto avaria & 95,67\% & 4,33\% \\ \hline
        cancelamento & 85,48\% & 14,52\% \\ \hline
        atraso & 51,19\% & 48,81\% \\ \hline
        culpa exclusiva do consumidor & 98,39\% & 1,61\% \\ \hline
        fechamento do aeroporto & 98,64\% & 1,36\% \\ \hline
        no show & 96,43\% & 3,56\% \\ \hline
        overbooking & 96,86\% & 3,14\% \\ \hline
        assistência cia aérea & 83,53\% & 16,47\% \\ \hline
        hipervulnerável & 96,10\% & 3,90\% \\ \hline
    \end{tabular}
    \caption{Distribuição dos features na base de dados manualmente extraída}
    \fonte{Elaborado pelo autor}
    \label{tab:features_distribution}
\end{table}

No projeto Concil-IA em si, há um processo de extração automatizada de dados (\citeonline{pereira2025using}), em quantidade descrita no Quadro \ref{tab:extraction_modes} que entretanto não pertine ao escopo deste trabalho, com o modelo de referência sendo o treinado exclusivamente com as sentenças extraídas manualmente.

\begin{table}[!ht]
    \centering
    \begin{tabular}{|| m{4 cm} | m{4 cm} ||}
        \hline
        Modo de Extração & Número de Entradas \\ \hline
        \hline
        \hline
        Manual & 1174 \\ \hline
        Automática & 687 \\ \hline
    \end{tabular}
    \caption{Proporção de modalidades de extração}
    \fonte{Elaborado pelo Autor}
    \label{tab:extraction_modes}
\end{table}

\subsection{Ferramentas Computacionais}
A implementação das etapas de pré-processamento, modelagem e explicabilidade foi realizada utilizando a linguagem de programação Python (3.13), com um conjunto de bibliotecas especializadas, conforme detalhado no Quadro \ref{tab:python_libs}

\begin{table}[!ht]
    \centering
    \begin{tabular}{|| m{3 cm} | m{5 cm}| m{5 cm} ||}
        \hline
        Categoria & Ferramentas/Bibliotecas & Usabilidade Principal \\
        \hline
        \hline
        Manipulação e tratamento de dados & pandas, numpy,
        imblearn.over\_sampling .RandomOverSampler & Limpeza, transformação,
        balanceamento de dados \\ \hline
        Modelagem preditiva & scikit-learn (DecisionTreeRegressor,
        RandomForestRegressor, AdaBoostRegressor, train\_test\_split) &
        Construção, avaliação e seleção de modelos de regressão \\ \hline
        Visualização e exportação & matplotlib, graphviz, tree.plot\_tree,
        tree.export\_graphviz, joblib & Geração de gráficos e visualização de
        árvores de decisão \\ \hline
        Avaliação de desempenho & scikit-learn (metrics.classification\_report),
        RMSE, MAE & Medição de desempenho dos modelos \\ \hline
        Explicabilidade (XAI) & shap (shap.Explainer, shap.plots.waterfall,
        shap.plots.bar) & Interpretação e explicação dos resultados dos
        modelos \\ \hline
        Utilitários e suporte & joblib, json, tqdm, packaging, python-dateutil,
        pytz, typing\_extensions, tzdata & Suporte à execução e ao ambiente
        computacional \\ \hline
    \end{tabular}
    \caption{Bibliotecas python utilizadas}
    \fonte{Elaborado pelo autor}
    \label{tab:python_libs}
\end{table}

As versões, uso específico, detalhamento e bibliotecas adicionais encontram-se no Apêndice \ref{tab:python_versions}

\section{Pipeline de Desenvolvimento do Modelo}

\subsection{Pré-processamento e Engenharia de Atributos}
Como detalhado na seção \ref{ch:dataset}, a base de dados pela qual o modelo se guiará foi extraída de sentenças jurídicas (anonimizadas) de processo a companhias aéreas para tabelas de dados estruturados.

Os \textit{features} de intervalo, tanto para extravio de bagagem quanto atraso de vôo, foram também discretizadas conforme os Quadros \ref{tab:faixas-intervalo-de-extravio} e \ref{tab:faixas-intervalo-de-atraso}

\begin{table}
    \centering
    \begin{tabular}{|| m{1.5 cm} | m{3 cm} | m{4.5 cm} ||}
        \hline
        Faixa &
        Intervalo do Extravio (Horas) &
        Frequência na Base de Dados
        Manualmente Extraída \\
        \hline
        \hline
        0 & 0             & 86,59 \% \\ \hline
        1 & 1 - 24      & 3,06 \% \\ \hline
        2 & 25 - 72   & 4,57 \% \\ \hline
        3 & 73 - 168 & 3,23 \% \\ \hline
        4 & 169         & 2,55 \% \\ \hline
    \end{tabular}
    \caption{Faixas de Intervalo de Extravio}
    \fonte{Elaborado pelo autor}
    \label{tab:faixas-intervalo-de-extravio}
\end{table}
\begin{table}[!ht]
    \centering
    \begin{tabular}{|| | m{4 cm} | m{4 cm} | m{4 cm} ||}
        \hline
        Faixa & Intervalo do Atraso (HH:MM) & Frequência na Base de Dados Manualmente Extraída \\
        \hline
        \hline
        -1 & Consumidor não chegou ao destino & 14,52 \% \\ \hline
        0 & 0 & 50,25 \% \\ \hline
        1 & 0:01 - 3:59 & 2,38 \% \\ \hline
        2 & 4:00 - 7:59:00 & 10,70 \% \\ \hline
        3 & 8:00 - 11:59 & 10,78 \% \\ \hline
        4 & 12:00 - 15:59 & 7,56 \% \\ \hline
        5 & 16:00 - 23:59 & 6,03 \% \\ \hline
        6 & 24:00 - 27:59 & 8,74 \% \\ \hline
        7 & 28:00 & 3,56 \% \\ \hline
    \end{tabular}
    \caption{Faixas utilizadas no feature intervalo de atraso}
    \fonte{Elaborado pelo autor}
    \label{tab:faixas-intervalo-de-atraso}
\end{table}


Inicialmente, realizou-se uma simples simplificação dos dados estruturados.
\textit{Features} categóricos foram convertidos em numéricos mediante substituição binária --- "Sim" ou "S" é substituído por '1', 'Não' ou "N" por '0', valores não identificados por '0' também ---, ao passo que atributos contínuos, como os intervalos temporais mencionados acima, foram discretizados em faixas baseadas na distribuição observada, usando o método dos quartis (\cite{pinheiro2009estatistica}).

% [TODO 2: Feature Selection virá aqui]
\cite{kuhn2013applied}

Depois, colunas com \textit{confactors} (que anulam o dano moral) foram também removidas, junto aos casos improcedentes.
--- Essa decisão foi tomada por observar-se que valores nulos prejudicam a acurácia e confundem o modelo ---
Portanto, os \textit{features} "culpa exclusiva do consumidor" e "condições climáticas/fechamento do aeroporto" são verificadas na interface do \textit{website} e não no modelo. --- Ou seja, antes do usuário poder inserir os fatores específicos do seu caso, ele é questionado sobre as condições climáticas do momento de seu vôo e se os problemas não decorrem de suas ações. Caso responda sim, é informado diretamente que a indenização por dano moral provavelmente será nula e desencorajado de usar o modelo.

O \textit{feature} "assistência da companhia aérea" (se a companhia aérea prestou ajuda efetiva para mitigar transtornos ao consumidor), que também seria um \textit{factor} negativo para o valor de indenização (embora não o anule) originalmente, e por isso teve sua interpretação e valores invertidos. Ele tornou-se o \textit{feature} "desamparo", presente apenas quando a companhia aérea falha em prestar auxílio efetivo ao consumidor. Dessa forma, todos os \textit{features} selecionados contribuem para o aumento do valor estimado.

Os \textit{features} "extravio temporário de bagagem" (binário) e "intervalo de extravio de bagagem" foram combinados (se não houve extravio de bagagem, o intervalo é 0). Da mesma forma, "atraso" e "intervalo de atraso" foram combinados, e depois o \textit{feature} "cancelamento/alteração de destino" também foi combinada em "intervalo de atraso", tornando-se a faixa '-1', para representar infinito (pois o consumidor nunca chegou ao destino contratado).

Após isso, os atributos redundantes, como os \textit{features} binários "extravio temporário de bagagem" e "atraso" foram removidos, bem como "cancelamento/alteração de destino". Isso deve-se a sua informação já ser contida em "intervalo de extravio de bagagem" e "intervalo de atraso", e contribui para simplificar o modelo final.

Por fim, Outliers (\cite{hawkins1980identification}) foram identificados pelo método dos quantis \cite{wilcox2012introduction} e tratados com remoção seletiva.
Para mitigar desbalanceamentos, aplicou-se a técnica \textit{RandomOverSampler} (da biblioteca \textit{Python} \textit{imblearn}), com a estratégia "\textit{not-majority}" para uniformizar as categorias-alvo (14, decididas baseado na distribuição de valores alvo \ref{fig:value_distribution}).

\begin{figure}[h]
    \centering
    \includegraphics[width=0.5\linewidth]{images/dano-moral-distribuicao.png}
    \caption{Distribuição de Valores para o Dano moral}
    \label{fig:value_distribution}
    \fonte{Elaborado pelo Autor}
\end{figure}

\subsection{Treinamento e Seleção do Modelo}
O conjunto de dados foi dividido em 80\% para treinamento e 20\% para teste, utilizando uma estratégia de amostragem estratificada para garantir que a distribuição das classes de valor fosse preservada em ambas as amostras. 

% [TODO 2: Depois de mudar a metodologia para incluir um conjunto de validação, citar o Machine Yearning]
Foram treinados e avaliados três algoritmos de regressão da biblioteca \textit{scikit-learn}: \textit{DecisionTreeRegressor} (\cite{blockeel2023decision}), \textit{RandomForestRegressor} (\cite{zhang2023compare}) e \textit{AdaBoostRegressor} (\cite{airlangga2024anomaly})

A otimização dos hiperparâmetros de cada modelo foi realizada por meio de uma abordagem heurística, com a variação de um hiperparâmetro por vez, devido às limitações de recursos computacionais disponíveis.

\subsection{Avaliação de Desempenho}
\label{sub:desempenho}
O desempenho dos modelos foi avaliado quantitativamente por meio de duas métricas principais: o Erro Quadrático Médio Raiz (RMSE), escolhido por sua capacidade de penalizar erros maiores de forma mais significativa, e o Erro Absoluto Médio (MAE), utilizado para fornecer uma estimativa da margem de erro média em termos práticos, (\cite{zhang2023compare}).
Além do desempenho preditivo, a seleção do modelo final considerou fatores secundários, como o custo computacional (memória e tempo de processamento) e a interpretabilidade inerente de cada algoritmo, sendo este último um critério fundamental para os objetivos do projeto.
Nesse contexto, com os 3 modelos possuindo desempenho similar, os fatores secundários foram determinantes na escolha final.

\subsection{Implementação da Explicabilidade (XAI)}
Para garantir a transparência das predições, foi implementado o \textit{framework} SHAP (\textit{SHapley Additive exPlanations} (\cite{lundberg2017unified})). Utilizando o método \textit{Explainer}, foram calculados os valores de Shapley para cada atributo de cada predição, quantificando a contribuição individual de cada fator para o resultado final.

Esses valores foram então utilizados para gerar visualizações, como gráficos de cascata (\textit{waterfall plots}) para explicações locais e gráficos de barras para a importância global dos atributos, traduzindo os resultados do modelo em justificativas compreensíveis para o usuário final.