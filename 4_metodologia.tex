\chapter{Metodologia}
\label{ch:metodologia}
O desenvolvimento deste trabalho seguiu uma abordagem quantitativa para o modelo e qualitativa para a explicabilidade;
com a aplicação de técnicas de Ciência de Dados e Aprendizado de Máquina
para a construção de um modelo preditivo de regressão e técnicas de Inteligência Artificial Explicável
para permitir a explicação e interpretação de dados por usuários leigos.

O processo metodológico do Projeto Concil-IA como um todo\ foi estruturado em um \textit{pipeline} de múltiplas etapas como observa-se na figura \ref{fig:concilia_pipeline}

\begin{figure}[ht]
    \centering
    \includegraphics[width=0.5\linewidth]{images/Concil-IA-Pipeline.png}
    \caption{Pipeline Concil-IA}
    \fonte{Elaborado pelo Autor}
    \label{fig:concilia_pipeline}
\end{figure}

No escopo específico deste trabalho, o foco recai sobre as etapas de processamento, modelagem e explicabilidade, conforme destacado no fluxo da Figura \ref{fig:project-pipeline}
Já este trabalho contempla apenas as etapas:
\begin{figure}[ht]
    \centering
    \includegraphics[width=0.5\linewidth]{images/Project-Pipeline.png}
    \caption{Pipeline desse projeto}
    \fonte{Elaborado pelo Autor}
    \label{fig:project-pipeline}
\end{figure}

As seções subsequentes detalham os recursos utilizados e as etapas percorridas, partindo da origem dos dados jurídicos (\ref{sec:base_dados}), passando pelas ferramentas computacionais empregadas, e culminando no fluxo de engenharia de atributos e treinamento dos modelos inteligentes

\section{Base de Dados e Ferramentas Computacionais}
\label{sec:base_dados}
A fundamentação empírica deste estudo reside em um conjunto de sentenças judiciais reais e no ecossistema de bibliotecas da linguagem Python, escolhidos para garantir a reprodutibilidade e a robustez das análises.

A seguir, detalha-se a composição do \textit{dataset} e o ambiente de desenvolvimento.

\subsection{Base de Dados}
\label{sub:dataset}
A base de dados é composta por 1.174 sentenças proferidas pelo Juizado Especial Cível da UFSC, versando especificamente sobre falhas na prestação de serviços de transporte aéreo à luz do Código de Defesa do Consumidor.

Os dados, anonimizados em conformidade com a Lei Geral de Proteção de Dados (LGPD), foram estruturados e extraídos para arquivos no formato CSV (\textit{comma-separated values}).
Após a anonimização, que garantiu a supressão de identificadores sensíveis das partes, procedeu-se à extração das variáveis (ou \textit{features}) relevantes para a quantificação do dano.
A tabela \ref{tab:feature_desc} apresenta as variáveis estruturadas utilizadas neste estudo.

\begin{table}[ht]
    \centering
    \begin{tabular}{|| m{6 cm} | m{10 cm} ||}
        \hline
        Feature & Descrição \\
        \hline
        \hline
        direito de arrependimento & O consumidor tentou cancelar ou alterar a
        compra no prazo de 7 dias, mas não lhe foi permitido fazê-lo. \\ \hline
        descumprimento de oferta & A empresa fornecedora não cumpriu com o
        ofertado ao consumidor(a), seja no valor da passagem, seja no assento no
        avião ou situações correlatas. \\ \hline
        extravio definitivo & Uma ou mais bagagens foram extraviadas e nunca
        foram recuperadas. \\ \hline
        extravio temporário & Uma ou mais bagagens foram extraviadas e
        posteriormente recuperadas. \\ \hline
        intervalo extravio temporário & O tempo decorrido até recuperar a
        bagagem, caso haja extravio temporário \\ \hline
        violação, furto ou avaria & Adulteração da bagagem ou um item dela \\ \hline
        cancelamento/alteração de destino & O consumidor não foi levado ao
        destino inicial \\ \hline
        atraso de vôo & O consumidor foi levado ao destino inicial, mas com
        atraso maior que 4 horas \\ \hline
        intervalo atraso & O tempo decorrido até o consumidor chegar ao seu
        destino, caso haja atraso de vôo \\ \hline
        culpa exclusiva consumidor & Todos os problemas relatados decorrem de
        ações ou inações do consumidor. \\ \hline
        condições climáticas / fechamento & Evento imprevisível e inevitável que
        comprovadamente impediu o aeroporto de operar. \\ \hline
        no show & Cancelamento automático do vôo de retorno em razão exclusiva
        do não comparecimento à viagem de ida, sem concordância do
        consumidor. \\ \hline
        overbooking & Venda de assentos em quantidade superior a comportada pela
        aeronave fazendo com que o consumidor não embarcasse. \\ \hline
        assistência cia aérea & A companhia aérea forneceu auxílio concreto ao
        consumidor para enfrentar o problema de atraso de voo ou cancelamento
        sem realocação/alteração de destino, e essa assistência inclui
        hospedagem, alimentação, ou transporte alternativo. \\ \hline
        hipervulnerável & O consumidor era idoso, gestante, possuía deficiência,
        fazia uso de medicamentos ou estava acompanhado de criança (entende-se
        criança a(o) infante que tem até 12 anos incompletos). \\ \hline
    \end{tabular}
    \caption{Descrição das Features}
    \fonte{Elaborado pelo autor}
    \label{tab:feature_desc}
\end{table}

Ressalta-se que, embora o projeto Concil-IA possua iniciativas de extração automatizada de dados \cite{pereira2025using}, conforme descrito na tabela \ref{tab:extraction_modes}, este trabalho utiliza exclusivamente o conjunto de dados extraído e validado manualmente (\textit{Gold Standard}), visando minimizar ruídos decorrentes de erros de interpretação automática de texto nesta fase de validação do modelo.

\begin{table}[!ht]
    \centering
    \begin{tabular}{|| m{4 cm} | m{4 cm} ||}
        \hline
        Modo de Extração & Número de Entradas \\ \hline
        \hline
        \hline
        Manual & 1174 \\ \hline
        Automática & 687 \\ \hline
    \end{tabular}
    \caption{Proporção de modalidades de extração}
    \fonte{Elaborado pelo Autor}
    \label{tab:extraction_modes}
\end{table}

A tabela \ref{tab:features_distribution} mostra a distribuição de valores para cada feature encontrada na base de dados manualmente extraída.

\begin{table}[!ht]
    \centering
    \begin{tabular}{|| m{6 cm} | m{2 cm} | m{2 cm} ||}
        \hline
        Features: & N & S \\ \hline
        \hline
        \hline
        direito de arrependimento & 98,65\% & 1,35\% \\ \hline
        descumprimento de oferta & 99,15\% & 0,85\% \\ \hline
        extravio definitivo & 96,86\% & 3,14\% \\ \hline
        extravio temporário & 86,84\% & 13,16\% \\ \hline
        violação furto avaria & 95,67\% & 4,33\% \\ \hline
        cancelamento & 85,48\% & 14,52\% \\ \hline
        atraso & 51,19\% & 48,81\% \\ \hline
        culpa exclusiva do consumidor & 98,39\% & 1,61\% \\ \hline
        fechamento do aeroporto & 98,64\% & 1,36\% \\ \hline
        no show & 96,43\% & 3,56\% \\ \hline
        overbooking & 96,86\% & 3,14\% \\ \hline
        assistência cia aérea & 83,53\% & 16,47\% \\ \hline
        hipervulnerável & 96,10\% & 3,90\% \\ \hline
    \end{tabular}
    \caption{Distribuição dos features na base de dados manualmente extraída}
    \fonte{Elaborado pelo autor}
    \label{tab:features_distribution}
\end{table}

\subsection{Ferramentas Computacionais}

A implementação do \textit{pipeline} de Ciência de Dados foi realizada na linguagem Python (versão 3.13), selecionada por sua ampla comunidade e suporte a bibliotecas de Aprendizado de Máquina. A tabela \ref{tab:python_libs} resume as principais bibliotecas empregadas.

\begin{table}[!ht]
    \centering
    \begin{tabular}{|| m{3 cm} | m{5 cm}| m{5 cm} ||}
        \hline
        Categoria & Ferramentas/Bibliotecas & Usabilidade Principal \\
        \hline
        \hline
        Manipulação e tratamento de dados & pandas, numpy,
        imblearn.over\_sampling .RandomOverSampler & Limpeza, transformação,
        balanceamento de dados \\ \hline
        Modelagem preditiva & scikit-learn (DecisionTreeRegressor,
        RandomForestRegressor, AdaBoostRegressor, train\_test\_split) &
        Construção, avaliação e seleção de modelos de regressão \\ \hline
        Visualização e exportação & matplotlib, graphviz, tree.plot\_tree,
        tree.export\_graphviz, joblib & Geração de gráficos e visualização de
        árvores de decisão \\ \hline
        Avaliação de desempenho & scikit-learn (metrics.classification\_report),
        RMSE, MAE & Medição de desempenho dos modelos \\ \hline
        Explicabilidade (XAI) & shap (shap.Explainer, shap.plots.waterfall,
        shap.plots.bar) & Interpretação e explicação dos resultados dos
        modelos \\ \hline
        Utilitários e suporte & joblib, json, tqdm, packaging, python-dateutil,
        pytz, typing\_extensions, tzdata & Suporte à execução e ao ambiente
        computacional \\ \hline
    \end{tabular}
    \caption{Bibliotecas python utilizadas}
    \fonte{Elaborado pelo autor}
    \label{tab:python_libs}
\end{table}

As versões, uso específico, detalhamento e bibliotecas adicionais encontram-se no Apêndice \ref{tab:python_versions}

\section{Pipeline de Desenvolvimento do Modelo}

O desenvolvimento do modelo preditivo seguiu um fluxo iterativo composto por três grandes fases
\begin{enumerate}
    \item O pré-processamento dos dados brutos,
    \item O treinamento e seleção de algoritmos,
    \item A implementação da camada de explicabilidade.
\end{enumerate}

Cada uma destas fases envolveu decisões de projeto visando equilibrar a precisão estatística com a coerência jurídica.

\subsection{Pré-processamento e Engenharia de Atributos}

Como detalhado na seção \ref{sub:dataset}, a base de dados pela qual o modelo se guiará foi extraída de sentenças jurídicas (anonimizadas) de processo a companhias aéreas para tabelas de dados estruturados.

Inicialmente, realizou-se a discretização das variáveis contínuas de tempo (atraso e extravio de bagagem), conforme os intervalos tipicamente utilizados na jurisprudência, apresentados nas tabelas \ref{tab:faixas-intervalo-de-extravio} e \ref{tab:faixas-intervalo-de-atraso}.

\begin{table}
    \centering
    \begin{tabular}{|| m{1.5 cm} | m{3 cm} | m{4.5 cm} ||}
        \hline
        Faixa &
        Intervalo do Extravio (Horas) &
        Frequência na Base de Dados
        Manualmente Extraída \\
        \hline
        \hline
        0 & 0             & 86,59 \% \\ \hline
        1 & 1 - 24      & 3,06 \% \\ \hline
        2 & 25 - 72   & 4,57 \% \\ \hline
        3 & 73 - 168 & 3,23 \% \\ \hline
        4 & 169         & 2,55 \% \\ \hline
    \end{tabular}
    \caption{Faixas de Intervalo de Extravio}
    \fonte{Elaborado pelo autor}
    \label{tab:faixas-intervalo-de-extravio}
\end{table}
\begin{table}[!ht]
    \centering
    \begin{tabular}{|| | m{4 cm} | m{4 cm} | m{4 cm} ||}
        \hline
        Faixa & Intervalo do Atraso (HH:MM) & Frequência na Base de Dados Manualmente Extraída \\
        \hline
        \hline
        -1 & Consumidor não chegou ao destino & 14,52 \% \\ \hline
        0 & 0 & 50,25 \% \\ \hline
        1 & 0:01 - 3:59 & 2,38 \% \\ \hline
        2 & 4:00 - 7:59:00 & 10,70 \% \\ \hline
        3 & 8:00 - 11:59 & 10,78 \% \\ \hline
        4 & 12:00 - 15:59 & 7,56 \% \\ \hline
        5 & 16:00 - 23:59 & 6,03 \% \\ \hline
        6 & 24:00 - 27:59 & 8,74 \% \\ \hline
        7 & 28:00 & 3,56 \% \\ \hline
    \end{tabular}
    \caption{Faixas utilizadas no feature intervalo de atraso}
    \fonte{Elaborado pelo autor}
    \label{tab:faixas-intervalo-de-atraso}
\end{table}


\paragraph{Tratamento de Variáveis e Filtros Lógicos}
Realizou-se a conversão de variáveis categóricas para formatos numéricos categóricos e a remoção estratégica de colunas.

\textit{Features} categóricos foram convertidos em numéricos mediante substituição binária 
--- "Sim" ou "S" é substituído por '1', 'Não' ou "N" por '0', valores não identificados por '0' também ---,
ao passo que atributos contínuos, como os intervalos temporais mencionados acima, foram discretizados em faixas baseadas na distribuição observada, usando o método dos quartis (\cite{pinheiro2009estatistica}).

Optou-se pela exclusão de casos improcedentes e de variáveis que atuam como "fatores de anulação" do dano moral.
Essa decisão justifica-se pois tais fatores são determinísticos: sua presença implica, juridicamente, a inexistência de dever de indenizar.

Portanto, os \textit{features} "culpa exclusiva do consumidor" e "condições climáticas/fechamento do aeroporto" são verificadas na interface do \textit{website} e não no modelo. --- Ou seja, antes do usuário poder inserir os fatores específicos do seu caso, ele é questionado sobre as condições climáticas do momento de seu vôo e se os problemas não decorrem de suas ações.
Caso responda sim, é informado diretamente que a indenização por dano moral provavelmente será nula e desencorajado de usar o modelo.

% [TODO 2: Feature Selection virá aqui]
% \cite{kuhn2013applied}
Adicionalmente, aplicou-se uma inversão lógica no atributo "assistência da companhia aérea". Originalmente um fator atenuante, ele foi transformado na variável "desamparo", que é contabilizada apenas quando a empresa falha em prestar auxílio.
Com isso, buscou-se uma monotonicidade positiva, onde a presença de qualquer \textit{feature} no vetor de entrada contribui positivamente para o aumento do valor predito.

Os \textit{features} "extravio temporário de bagagem" (binário) e "intervalo de extravio de bagagem" foram combinados (se não houve extravio de bagagem, o intervalo é 0). Da mesma forma, "atraso" e "intervalo de atraso" foram combinados, e depois o \textit{feature} "cancelamento/alteração de destino" também foi combinada em "intervalo de atraso", tornando-se a faixa '-1', para representar infinito (pois o consumidor nunca chegou ao destino contratado).

Variáveis redundantes ou correlatas foram fundidas ou removidas para reduzir a dimensionalidade.
Os \textit{features} binários "extravio temporário de bagagem" e "atraso" foram removidos, uma vez que sua informação já é contida nas variáveis de intervalo.

Já o \textit{Feature} "cancelamento de voo" foi incorporado à variável de tempo como um valor de atraso infinito (faixa '-1'), simplificando a interpretação do modelo.

\paragraph{Tratamento de Outliers e Balanceamento}
Valores discrepantes (\textit{outliers} \cite{hawkins1980identification}) foram identificados pelo método dos quantis \cite{wilcox2012introduction} e tratados com remoção seletiva.

Para mitigar o efeito de desbalanceamentos de classes, aplicou-se a técnica \textit{RandomOverSampler} (da biblioteca \textit{Python} \textit{imblearn}), com a estratégia "\textit{not-majority}" para uniformizar as categorias-alvo (14, decididas baseado na distribuição de valores alvo vistas na Figura \ref{fig:value_distribution}).

\begin{figure}[h]
    \centering
    \includegraphics[width=0.5\linewidth]{images/dano-moral-distribuicao.png}
    \caption{Distribuição de Valores para o Dano moral}
    \label{fig:value_distribution}
    \fonte{Elaborado pelo Autor}
\end{figure}

% [TODO 2: Corrigir o balanceamento no código e atualizar aqui]
Reconhece-se que aplicar o balanceamento antes de separar os conjuntos de treino e teste é uma falha metodológica, uma vez que introduz vieses se o conjunto de teste incluir dados replicados, e será corrigida futuramente.

\subsection{Treinamento e Seleção do Modelo}

O conjunto de dados foi dividido em 80\% para treinamento e 20\% para teste, utilizando uma estratégia de amostragem estratificada para garantir que a distribuição das classes de valor fosse preservada em ambas as amostras. 

% [TODO 2: Depois de mudar a metodologia para incluir um conjunto de validação, citar o Machine Yearning]
Nesta primeira etapa, foram treinados e avaliados três algoritmos de regressão da biblioteca \textit{scikit-learn}:
\textit{DecisionTreeRegressor} (\cite{blockeel2023decision}), 
\textit{RandomForestRegressor} (\cite{zhang2023compare}) 
e \textit{AdaBoostRegressor} (\cite{airlangga2024anomaly})

A otimização dos hiperparâmetros de cada modelo foi realizada por meio de uma abordagem heurística, com a variação de um hiperparâmetro por vez, devido às limitações de recursos computacionais disponíveis.

\subsection{Avaliação de Desempenho}
\label{sub:desempenho}
A validação do modelo não se restringiu à precisão numérica, incorporando também critérios de transparência algorítmica e performance computacional

Utilizou-se o Erro Quadrático Médio Raiz (RMSE) para penalizar grandes desvios
e o Erro Absoluto Médio (MAE) para mensurar a margem de erro média em valores monetários reais \cite{zhang2023compare}.

Além do desempenho preditivo, a seleção do modelo final considerou fatores secundários,
como o custo computacional (memória e tempo de processamento
e a interpretabilidade inerente de cada algoritmo,
sendo este último um critério fundamental para os objetivos do projeto.

Nesse contexto, com os 3 modelos possuindo desempenho similar, os fatores secundários foram determinantes na escolha final.

\subsection{Implementação da Explicabilidade (XAI)}
Para garantir a transparência das predições, foi implementado o \textit{framework} SHAP (\textit{SHapley Additive exPlanations} (\cite{lundberg2017unified})).

Utilizando o método \textit{Explainer}, o sistema calcula a contribuição marginal de cada fato jurídico (ex: tempo de atraso, extravio) para o valor final da indenização, gerando visualizações como o \textit{waterfall plot}.

Isso permite que o usuário compreenda não apenas "quanto" receberá, mas "por que" aquele valor foi sugerido.