\chapter{Fundamentação Teórica}
\label{ch:teorica}

Este capítulo apresenta os conceitos-chave que formam a base teórica deste trabalho. A abordagem parte do contexto jurídico dos meios de solução de conflitos (\ref{contexto_juridico}), avança para os fundamentos computacionais da Inteligência Artificial e do Aprendizado de Máquina (\ref{ai_ml}) e, por fim, detalha as técnicas de regressão (\ref{regression}) e explicabilidade (\ref{xai} utilizadas no desenvolvimento do modelo preditivo.

\section{Meios Autocompositivos e Resolução de Disputas Online}
\label{contexto_juridico}

O sistema judiciário brasileiro, diante do crescente volume de processos, tem incentivado a adoção de métodos autocompositivos para a solução de litígios. Dentre eles, destaca-se a conciliação.

\subsection{Conciliação no Contexto Brasileiro}
A conciliação é um método no qual um terceiro, neutro e imparcial — o conciliador —, facilita o diálogo entre as partes para que elas possam construir, por si mesmas, uma solução para o conflito.
O objetivo é alcançar um acordo que satisfaça os interesses de ambos os envolvidos, de forma mais célere e menos adversarial que o processo judicial tradicional.
A promoção desses meios é uma política oficial do Conselho Nacional de Justiça (CNJ), formalizada pela Resolução nº 125 de 2010 \cite{RCNJ1252010}, para evitar litígios desnecessários. Litigio refere-se à propensão de conflitos recorrerem a decisão judicial.

\subsection{Resolução de Disputas Online (ODR)}
A Resolução de Disputas Online, ou ODR (do inglês, \textit{Online Dispute Resolution}), "consiste  na  utilização  da  tecnologia  da  informação  e  da  comunicação  no  processo  de  solução  de  conflitos,  seja  na  totalidade  do  procedimento  ou  somente  em  parte  deste. 
Dentre  os  procedimentos  que  podem  adotar  o  modelo  da  ODRs,  estão  a  arbitragem,  a  mediação,  a  conciliação  ou  a  negociação,  que  o  fazem  por  intermédio  de  ferramentas  automatizadas  (total  ou  parcialmente)" \cite{lima2016online}.
As plataformas de ODR transferem o processo de negociação para o ambiente digital, oferecendo ferramentas que podem ampliar o acesso à justiça e otimizar a comunicação entre as partes, independentemente de barreiras geográficas.
O projeto Concil-IA se insere nesse contexto, utilizando a tecnologia para apoiar o processo de conciliação.

\section{Inteligência Artificial e Aprendizado de Máquina}
\label{ai_ml}

A Inteligência Artificial (IA) é um vasto campo da ciência da computação dedicado à criação de sistemas capazes de realizar tarefas que, historicamente, exigiriam inteligência humana, como percepção visual, reconhecimento de fala, tomada de decisão e tradução de idiomas.

Contemporaneamente, \citeonline{russell1995modern}, definem a IA não apenas pela simulação do pensamento humano, mas pela construção de "agentes racionais": sistemas que percebem seu ambiente e agem de forma a maximizar suas chances de sucesso na realização de um objetivo.

Já \citeonline{kaplan2019siri} oferecem uma definição operacional mais alinhada à ciência de dados moderna, descrevendo a IA como a capacidade de um sistema de "interpretar corretamente dados externos, aprender a partir desses dados e utilizar esse aprendizado para atingir objetivos específicos e tarefas através de adaptação flexível".


É importante distinguir, no escopo deste trabalho, os dois grandes tipos de IA classificados pela literatura \cite{searle1980minds}:
\begin{itemize}
    \item \textbf{IA Forte (General AI):} Sistemas hipotéticos que possuiriam consciência e capacidade de raciocínio generalista equivalente ou superior à humana.
    \item \textbf{IA Fraca (Narrow AI):} Sistemas projetados para resolver problemas específicos e bem delimitados, como reconhecimento de imagem, tradução automática ou, no caso deste estudo, a predição de valores monetários em disputas jurídicas.
\end{itemize}

O modelo que este estudo busca desenvolver enquadra-se na categoria de IA Fraca, utilizando métodos estatísticos para identificar padrões em decisões passadas sem, contudo, exercer julgamento moral ou subjetivo.

\subsection{\textit{Machine Learning}}
\label{sec:machine_learning}
Dentre as subáreas da Inteligência Artificial, o Aprendizado de Máquina (\textit{Machine Learning}) destaca-se como a abordagem dominante para a construção de sistemas preditivos.

Enquanto a IA simbólica tradicional dependia de regras explicitamente programadas (estruturas \textit{if-then} manuais), o \textit{Machine Learning} foca no desenvolvimento de algoritmos que permitem aos computadores aprenderem essas regras empiricamente.

Com isso, máquinas tornaram-se capazes de aprender a partir de dados e criar modelos especializados em fazer previsões ou realizar tomadas de decisões com base em novas observações \cite{zhou2021machine}.

Conforme definido formalmente por \citeonline{mitchell1997machine}, diz-se que um programa de computador aprende a partir da experiência $E$ com respeito a uma classe de tarefas $T$ e medida de desempenho $D$, se seu desempenho nas tarefas em $T$, medido por $D$, melhora com a experiência $E$.

No contexto deste trabalho:
\begin{itemize}
    \item \textbf{Tarefa ($T$):} Prever o valor de acordos ou indenizações;
    \item \textbf{Experiência ($E$):} O conjunto de dados históricos de processos anteriores;
    \item \textbf{Desempenho ($D$):} A precisão da predição em relação ao valor real (medida por métricas de erro).
    \cite{mitchell1997machine}
\end{itemize}

Existem diferentes paradigmas de aprendizado, sendo os principais: supervisionado, não-supervisionado e por reforço.

Este trabalho utiliza a abordagem de \textbf{aprendizado supervisionado}, na qual o modelo é treinado com um conjunto de dados rotulado, ou seja, onde as "respostas corretas" (o valor final da conciliação) são fornecidas ao algoritmo durante o treinamento \cite{russell2021artificial}.

Para garantir a robustez do modelo, o processo de construção envolve a divisão dos dados em conjuntos de treinamento e teste.
O conjunto de treinamento permite ao modelo ajustar seus parâmetros internos para minimizar o erro nas predições, enquanto o conjunto de teste, composto por dados nunca vistos pelo modelo, avalia sua capacidade de generalização \cite{zhou2021machine}.

Dois desafios comuns nessa etapa são o viés (\textit{bias}) e o \textit{overfitting}.
O viés ocorre quando o modelo faz suposições simplistas demais, falhando em capturar padrões relevantes (subajuste).
Já o \textit{overfitting} (sobreajuste) acontece quando o modelo "decora" o ruído dos dados de treinamento, apresentando desempenho excelente nos dados conhecidos, mas falhando drasticamente em novos casos \cite{zhou2021machine}.
A escolha de algoritmos como Árvores de Decisão, detalhados a seguir, busca equilibrar essa relação entre viés e variância.

\section{Modelagem Preditiva com Regressão}
\label{regression}

A análise preditiva utiliza técnicas de aprendizado de máquina para fazer previsões sobre eventos futuros com base em dados históricos.
No domínio jurídico, isso pode se traduzir na previsão de desfechos de casos ou, como neste estudo, na estimativa de valores de indenização.

\subsection{Regressão}
A regressão é uma tarefa de aprendizado supervisionado cujo objetivo é prever um valor de saída contínuo e numérico. Diferentemente da classificação, que prevê uma categoria (e.g., "procedente" ou "improcedente"), a regressão estima uma quantidade, como o valor monetário de uma indenização por dano moral.

\subsection{Árvores de Decisão para Regressão}
Uma Árvore de Decisão (\textit{Decision Tree}) é um algoritmo de aprendizado de máquina que constrói um modelo preditivo na forma de uma estrutura de árvore.
Cada nó interno da árvore representa uma pergunta sobre um dos atributos do caso (e.g., "O atraso do voo foi maior que 4 horas?"), e cada nó folha representa uma predição de valor
Uma das principais vantagens deste algoritmo é sua alta interpretabilidade, pois o fluxo de decisões que leva a uma previsão pode ser facilmente visualizado e compreendido.
O modelo desenvolvido inicialmente neste trabalho, o \textit{DecisionTreeRegressor}, é uma implementação deste conceito para tarefas de regressão.

\section{Inteligência Artificial Explicável (XAI)}
\label{xai}

A crescente utilização de modelos de IA em áreas críticas como o Direito levanta a questão da transparência.
Muitos algoritmos complexos funcionam como uma "caixa-preta" (\textit{black box}), tornando difícil para um ser humano compreender como uma decisão específica foi alcançada.
A Inteligência Artificial Explicável (XAI) é o campo de pesquisa que busca desenvolver técnicas para tornar os modelos de IA mais transparentes e interpretáveis.

\subsection{SHapley Additive exPlanations (SHAP)}
O \textit{SHAP} é um \textit{framework} unificado para a interpretação de modelos preditivos, baseado no conceito de valores de \textit{Shapley} da teoria dos jogos cooperativos \cite{salih2025perspective}.
Ele quantifica a contribuição de cada fator (ou \textit{feature}) do caso para o resultado da predição. Para uma determinada previsão, o \textit{SHAP} calcula como cada informação (e.g., o tempo de atraso do voo, a ocorrência de extravio de bagagem) empurrou o valor final para cima ou para baixo em relação a um valor base (a média das predições).
Isso permite gerar explicações tanto para casos individuais (explicações locais) quanto para o comportamento geral do modelo (explicações globais), sendo uma ferramenta poderosa para garantir a transparência e a confiabilidade das decisões algorítmicas.