\section{Trabalhos Futuros}

As limitações encontradas abrem caminhos para futuras pesquisas.

A principal recomendação é a expansão contínua da base de dados, incorporando um número maior e mais variado de sentenças judiciais.

Com um conjunto de dados mais robusto, sugere-se a exploração de modelos mais complexos, como Redes Neurais, que podem capturar padrões mais sutis nos dados.

Adicionalmente, propõe-se a aplicação de técnicas mais avançadas de otimização de hiperparâmetros, como \textit{Grid Search} ou \textit{Bayesian Optimization}, que se tornariam computacionalmente viáveis em um ambiente com mais recursos.

Por fim, uma vez que o modelo final esteja integrado à plataforma online, um trabalho futuro de grande valor seria a coleta e análise de feedback dos usuários finais para refinar e aprimorar a usabilidade e a precisão da ferramenta em um contexto de uso real.

% [TODO 2: Fortalecer a metodologia.]
% Por exemplo, ao invés de apenas apresentar as métricas de RMSE e MAE, poderia adicionar uma subseção de "Análise Estatística dos Resultados", onde aplicaria testes para verificar se as features que o SHAP apontou como mais importantes possuem distribuições estatisticamente diferentes nos casos de acerto e erro do modelo, assim como faz o exemplo. Isso agregaria um grande valor analítico e de rigor ao seu trabalho]