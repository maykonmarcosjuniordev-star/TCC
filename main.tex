%dica: use a opção oneside se houver um limite (e.g., 20) de páginas
\documentclass[embeddedlogo]{ufsc-thesis-rn46-2019}
\usepackage[T1]{fontenc} % fontes
\usepackage[utf8]{inputenc} % UTF-8
\usepackage{lipsum} % Gerador de texto
\usepackage{pdfpages} % Inclui PDF externo (ficha catalográfica)
\usepackage{multirow}
\usepackage{colortbl}
\usepackage{pgfplots}
\usepackage{tocbibind}
\usepackage{subcaption}
\usepackage{afterpage}
\usepackage{listings}
\usepackage{graphicx}

%%%%%%%%%%%%%%%%%%%%%%%%%%%%%%%%%
%%% Configurações da classe (dados do trabalho)                  %%%
%%%%%%%%%%%%%%%%%%%%%%%%%%%%%%%%%
% Preâmbulo
\titulo{Uso de Modelos de Inteligência Artificial para Predição de Julgamentos}
\autor{Maykon Marcos Junior}
% Importante! Para documentos em inglês, não use today, digite a data em
% pt_BR, como deve aparecer na folha de certificação.
% [TODO 2: Trocar para a data de TCC 2]
\data{08/12/2025}
\instituicao{Universidade Federal de Santa Catarina}
\centro{Centro Tecnológico}
% \programa{Programa de Graduação em Ciência da Computação}
% \tese % ou \dissertacao
\local{Florianópolis} % Apenas cidade! Sem estado
% template da BU usa doutor/mestre em minúsculo, Bacharel/Licenciado em Title case.
\titulode{bacharel em Ciência da Computação}

%%% Atenção! No caso de TCC, além de usar \tcc, outros comandos devem ser fornecidos:
%%%
\tcc
\departamento{Departamento de Informática e Estatística}
\curso{Ciência da Computação}
\titulode{Bacharel em Ciência da Computação}
% %% Para TCCs, orientadores e coorientadores podem ser externos, logo a
% %% BU exige que sua afiliação seja explicitada. Por padrão, assume-se
% %% UFSC. Você pode alterar a afiliação com os comandos abaixo:
% \afiliacaoorientador{Universidade Federal de Santa Catarina}
% \afiliacaocoorientador{Universidade Federal da Terra de Ninguém}

% Orientador, coorientador, membros da banca e coordenador
% As regras da BU agora exigem que Dr. apareça **depois** do nome
% Dica: para gerar Profᵃ. use Prof\textsuperscript{a}.
% Dica 2: para feminino use \orientadora e \coorientadora
\orientador{Prof. Jônata Tyska Carvalho, Dr.}
\coorientadora{Prof\textsuperscript{a}. Isabela Cristina Sabo, Dra.}
% \membrobanca{Prof\textsuperscript{a}. Isabela Cristina Sabo, Dra.}{Universidade Federal de Santa Catarina}
\membrobanca{Prof. Elder Rizzon Santos, Dr.}{Universidade Federal de Santa Catarina}
\membrobanca{Prof. Renato Fileto, Dr.}{Universidade Federal de Santa Catarina}
% Dica: se feminino, \coordenador
% \coordenador{Prof. ---, Dr.}

\pgfplotsset{compat=1.18}

% ---

% ---
% compila o indice
% ---
\makeindex


\begin{document}
\toggletrue{ufscthesistcc}
\begin{figure}[h] % 'h' positions the figure here
    \centering % Centers the figure
    \includegraphics[width=0.2\textwidth]{images/UFSC.png} % Adjust width as needed
   % \caption{} % Optional caption
    \label{fig:ufsc} % Optional label for referencing
\end{figure}

%%%%%%%%%%%%%%%%%%%%%%%%%%%%%%%
%%% Principais elementos pré-textuais                              %%%
%%%%%%%%%%%%%%%%%%%%%%%%%%%%%%%

% Inicia parte pré-textual do documento capa, folha de rosto, folha de
% aprovação, aprovação, resumo, lista de tabelas, lista de figuras, etc.
%\pretextual
\imprimircapa
\imprimirfolhaderosto*
% [TODO 2: Refazer ao final do projeto on https://ficha.bu.ufsc.br/]
\protect\incluirfichacatalografica{docs/ficha.pdf}


  % Aqui deve ser inserido um resumo de 150 a 500 palavras (limitação de tamanho dada pela BU).
  % A linguagem deve ser português e a hifenização já foi alterada.
  % O resumo em português deve preceder o resumo em inglês, mesmo que o trabalho seja escrito em inglês.
  % A BU também diz que deve ser usada a voz ativa e o discurso deve ser na 3ª pessoa.
  % A estrutura do resumo pode ser similar a estrutura usada em artigos: Contexto -- Problema -- Estado da arte -- Solução proposta  -- Resultados.
\begin{resumo}[Resumo]
O judiciário brasileiro enfrenta um crescente volume de litigiosidade, impulsionando a adoção de tecnologias de Inteligência Artificial (IA) para otimizar a resolução de conflitos.

Neste contexto, o presente trabalho, derivado das atividades desenvolvidas no projeto Concil-IA, teve como objetivo central o desenvolvimento  de um modelo preditivo de regressão para estimar valores de indenização por danos morais em disputas consumeristas de transporte aéreo, com foco em sua explicabilidade.

A metodologia envolveu o pré-processamento de uma base de dados com 1.174 sentenças judiciais, o treinamento e a avaliação de algoritmos de aprendizado de máquina, como o \textit{DecisionTreeRegressor}, e a aplicação da técnica de Inteligência Artificial Explicável (XAI) \textit{SHapley Additive exPlanations} (SHAP) para interpretar os resultados.

Como resultado, foi desenvolvido um modelo funcional com um Erro Quadrático Médio Raiz (RMSE) de R\$ 2.260,57 e um Erro Absoluto Médio (MAE) de R\$ 1.577,21, capaz de gerar predições de valor e identificar os fatores de maior impacto em cada decisão.
Conclui-se que a abordagem é tecnicamente viável e oferece um instrumento de auxílio para as partes em processos de conciliação, fornecendo estimativas transparentes e justificáveis que contribuem para a promoção da pacificação social e para o avanço dos sistemas de resolução de disputas online (ODR).
  \vspace{\baselineskip} 
  % Atenção! a BU exige separação através de ponto (.). Ela recomanda de 3 a 5 keywords
  \textbf{Palavras-chave:} Inteligência Artificial. Aprendizado de Máquina. Regressão. Explicabilidade. Resolução de Disputas Online.
\end{resumo}

\begin{abstract}
The Brazilian judiciary faces a increasing volume of litigation, driving the adoption of Artificial Intelligence (AI) technologies to optimize dispute resolution.

In this context, this work, derived from the activities developed in the Concil-IA project, had as its central objective the development of a predictive regression model to estimate compensation amounts for moral damages in air transport consumer disputes, with a focus on their explainability.
The methodology involved the preprocessing of a database with 1,174 court judgments, the training and evaluation of machine learning algorithms, such as \textit{DecisionTreeRegressor}, and the application of the Explainable Artificial Intelligence (XAI) technique \textit{SHapley Additive exPlanations} (SHAP) to interpret the output.
As a result, a functional model was developed with a Root Mean Square Error (RMSE) of R\$ 2,260,57 and a Mean Absolute Error (MAE) of R\$ 1,577,21 Reais, capable of generating value predictions and identifying the factors with the greatest impact on each decision.

The conclusion is that the approach is technically feasible and offers a tool to assist parties in conciliation proceedings, providing transparent and justifiable estimates that contribute to the promotion of social pacification and the advancement of online dispute resolution (ODR) systems.

  \vspace{\baselineskip} 
  \textbf{Keywords:} Artificial Intelligence. Machine Learning. Regression. Explainability. Online Dispute Resolution.
\end{abstract}


\listoffigures*
\listoftables*

\tableofcontents*

%%%%%%%%%%%%%%%%%%%%%%%%%%%
%%% Corpo do texto                                               %%%
%%%%%%%%%%%%%%%%%%%%%%%%%%%
\textual

\chapter{Introdução}

O Poder Judiciário brasileiro enfrenta um cenário de crescente litigiosidade, conforme aponta o relatório "Justiça em Números" do Conselho Nacional de Justiça  \cite{da2024relatorio}.
Em resposta a este desafio, o CNJ tem promovido duas agendas estratégicas principais: a primeira, consolidada pela Resolução nº 125/2010, foca na promoção de meios autocompositivos de solução de conflitos, como a conciliação e a mediação \cite{RCNJ1252010}
a segunda, impulsionada por resoluções como a de nº 332/2020 \cite{RCNJ3322020}, prioriza o uso de tecnologias e Inteligência Artificial (IA) para modernizar e otimizar os serviços judiciais, bem como a Resolução nº 615 de 2025, que estabelece diretrizes para o desenvolvimento, utilização e governança de soluções desenvolvidas com recursos de inteligência artificial no Poder Judiciário \cite{RCNJ6152025}.

Embora eficazes, as sessões de conciliação frequentemente encontram um obstáculo prático: a assimetria de expectativas entre as partes quanto ao valor da indenização por dano moral. Autores de processos, muitas vezes, possuem uma percepção superestimada do valor que lhes é devido, enquanto os réus tendem a oferecer quantias consideravelmente inferiores. Essa disparidade dificulta o consenso e, consequentemente, a celebração de acordos, prolongando o litígio.

% [TODO: Mudar isso para formato problema-hipótese-objetivos]
% Deve ser formatado como problema-hipótese → “É possível desenvolver um preditor satisfatório para casos de compensação de dano moral? Sim, de forma …”
Neste contexto, o projeto de pesquisa Concil-IA \cite{sabo2025avancos}, uma iniciativa multidisciplinar da Universidade Federal de Santa Catarina (UFSC) em parceria com o Juizado Especial Cível (JEC) da instituição, surge como uma resposta que integra ambas as agendas do CNJ. O projeto visa desenvolver um sistema de Resolução de Disputas Online (ODR, do inglês \textit{Online Dispute Resolution}) baseado em IA, com o objetivo de auxiliar na conciliação de conflitos consumeristas, especificamente em ações contra companhias aéreas.

O presente trabalho de conclusão de curso concentra-se no desenvolvimento de um modelo preditivo para estimar o valor da indenização por dano moral e na aplicação de técnicas de Inteligência Artificial Explicável (XAI) para conferir transparência aos seus resultados.
A partir de uma base de dados de 1.174 sentenças judiciais, foram utilizadas ferramentas de Aprendizado de Máquina (\textit{Machine Learning}), como a biblioteca \textit{Scikit-learn}, para treinar modelos de regressão, e o método \textit{SHapley Additive exPlanations} (SHAP) para interpretar as predições geradas \cite{salih2025perspective}.

Espera-se que o modelo resultante sirva como uma ferramenta de apoio para as partes e para os conciliadores, fornecendo uma estimativa fundamentada e transparente do valor de uma eventual condenação. Ressalta-se, contudo, que a solução é proposta como um instrumento de auxílio à tomada de decisão, e não como um substituto da análise e deliberação humana, respeitando-se o devido processo legal e a autonomia dos envolvidos.

\label{ch:intro}
\section{Objetivos}
\subsection{Objetivo geral}
\label{sub:objetivo_geral}
Desenvolver e implementar um modelo de Inteligência Artificial Explicável capaz de prever o valor de danos morais decididos em julgamentos relacionados a direitos do consumidor em voos comerciais, contribuindo para a explicabilidade de decisões judiciais e modelos de IA, e permitindo disponibilizar essa funcionalidade ao público por meio de uma interface web.
\subsection{Objetivos específicos}
\label{sub:objetivos_spec}
\begin{enumerate}[label=(\roman*)]
    \item Desenvolver e treinar um modelo de IA utilizando dados estruturados obtidos de sentenças jurídicas para prever valores de danos morais.
    \item Analisar diferentes algoritmos e abordagens para construção do modelo, avaliando métricas de desempenho e identificando razões para diferenças entre eles.
    \item Integrar o modelo a uma plataforma web já existente, permitindo que usuários consultem previsões de valores de danos morais com base em dados fornecidos.
    \item Compilar e documentar todo o processo, desde a obtenção e estruturação dos dados até a análise dos resultados e as etapas de integração ao site.
    \item Analisar a variabilidade dos resultados usando diferentes estratégias de divisão e formatação dos dados;
    \item Analisar a importância das \textit{features} do melhor modelo encontrado.
\end{enumerate}


De forma geral, esta metodologia concentra-se na aplicação de IA para auxiliar na predição de valores de danos morais em processos judiciais relacionados ao transporte aéreo, dentro do contexto de direitos do consumidor.
As etapas incluem:
\begin{enumerate}
    \item Processamento de aproximadamente 1.174 sentenças cedidas pelo Juizado de Pequenas Causas da UFSC.
    \item Criação de um \textit{dataset} estruturado com variáveis relevantes (e.g., extravio de bagagem, atraso).
    \item Avaliação de técnicas de \textit{feature selection} 
    \item Avaliação de técnicas de pré-processamento de dados
    \item Avaliação de algoritmos disponíveis na biblioteca \textit{Scikit-learn}
    \item Implementação de técnicas de explicabilidade utilizando o método \textit{SHAP}
    \item Geração de relatórios que documentem o processo, os resultados e as limitações do modelo.
\end{enumerate}

\section{Organização do trabalho}
Os próximos capítulos deste trabalho estão organizados da seguinte forma:
o Capítulo 2 consiste na fundamentação teórica, onde serão abordados conceitos fundamentais para a compreensão do trabalho, tais como inteligência artificial, explicabilidade, aprendizado de máquina e os diferentes modelos regressores que serão utilizados.
O Capítulo 3 aborda os trabalhos relacionados.
O Capítulo 4 explica a metodologia empregada para melhorar e avaliar os resultados.
O Capítulo 5 expõe e discute os resultados obtidos até então.
Por fim, o Capítulo 6 apresenta as conclusões retiradas deste trabalho, como também apontar os próximos passos do projeto.

Na elaboração dessa dissertação, modelos de linguagem como OpenAI Chat GPT, Gemini AI e Manus, foram utilizados para revisão e estruturação do texto.
\chapter{Fundamentação Teórica}
\label{ch:teorica}

Este capítulo apresenta os conceitos-chave que formam a base teórica deste trabalho. A abordagem parte do contexto jurídico dos meios de solução de conflitos (\ref{contexto_juridico}), avança para os fundamentos computacionais da Inteligência Artificial e do Aprendizado de Máquina (\ref{ai_ml}) e, por fim, detalha as técnicas de regressão (\ref{regression}) e explicabilidade (\ref{xai} utilizadas no desenvolvimento do modelo preditivo.

\section{Meios Autocompositivos e Resolução de Disputas Online}
\label{contexto_juridico}

O sistema judiciário brasileiro, diante do crescente volume de processos, tem incentivado a adoção de métodos autocompositivos para a solução de litígios. Dentre eles, destaca-se a conciliação.

\subsection{Conciliação no Contexto Brasileiro}
A conciliação é um método no qual um terceiro, neutro e imparcial — o conciliador —, facilita o diálogo entre as partes para que elas possam construir, por si mesmas, uma solução para o conflito. O objetivo é alcançar um acordo que satisfaça os interesses de ambos os envolvidos, de forma mais célere e menos adversarial que o processo judicial tradicional. A promoção desses meios é uma política oficial do Conselho Nacional de Justiça (CNJ), formalizada pela Resolução nº 125 de 2010 \cite{cnj2010relatorio}, para evitar litígio

% [? TODO: definir litígio].

\subsection{Resolução de Disputas Online (ODR)}
A Resolução de Disputas Online, ou ODR (do inglês, \textit{Online Dispute Resolution}), "consiste  na  utilização  da  tecnologia  da  informação  e  da  comunicação  no  processo  de  solução  de  conflitos,  seja  na  totalidade  do  procedimento  ou  somente  em  parte  deste.  Dentre  os  procedimentos  que  podem  adotar  o  modelo  da  ODRs,  estão  a  arbitragem,  a  mediação,  a  conciliação  ou  a  negociação,  que  o  fazem  por  intermédio  de  ferramentas  automatizadas  (total  ou  parcialmente)" \cite{lima2016online}. As plataformas de ODR transferem o processo de negociação para o ambiente digital, oferecendo ferramentas que podem ampliar o acesso à justiça e otimizar a comunicação entre as partes, independentemente de barreiras geográficas. O projeto Concil-IA se insere nesse contexto, utilizando a tecnologia para apoiar o processo de conciliação.

\section{Inteligência Artificial e Aprendizado de Máquina}
\label{ai_ml}

A Inteligência Artificial (IA) é um vasto campo da ciência da computação dedicado à criação de sistemas capazes de realizar tarefas que normalmente exigiriam inteligência humana, como percepção visual, reconhecimento de fala, tomada de decisão e tradução de idiomas.

% [TODO: Adicionar uma seção de IA]

\subsection{\textit{Machine Learning}}
\label{sec:machine_learning}

\textit{Machine learning} é um conjunto de técnicas que utiliza métodos computacionais e matemáticos para permitir que máquinas possam aprender a partir de dados e criar modelos especializados em fazer previsões ou realizar tomadas de decisões com base em novas observações. \cite{zhou2021machine}

O Aprendizado de Máquina (\textit{Machine Learning}) é um subcampo fundamental da IA.
Conforme definido por \citeonline{mitchell1997machine}, um programa de computador aprende a partir da experiência E com respeito a alguma classe de tarefas T e medida de desempenho D, se seu desempenho em tarefas em T, medido por D, melhora com a experiência E. Em outras palavras, em vez de ser explicitamente programado, o sistema "aprende" padrões diretamente dos dados. Este trabalho utiliza a abordagem de aprendizado supervisionado, na qual o modelo é treinado com um conjunto de dados onde as "respostas corretas" (rótulos) são fornecidas.

A fim de tornar o modelo mais preciso, existem diferentes abordagens dentro do escopo do \textit{machine learning} que podem ser utilizadas, a depender do tipo de dado que será usado (texto, imagem, vídeo, áudio, dentre outros tipos de mídia), bem como a tarefa para a qual o modelo foi designado. Por exemplo, existem modelos especializados em realizar a classificação de determinada característica de um conjunto de dados, no caso de modelos de classificação, ou em calcular a probabilidade de pertencer a essa característica, no caso de modelos de regressão. \cite{zhou2021machine}

Usualmente, em modelos preditores, o processo da construção do modelo envolve a divisão do conjunto de dados a ser usado em treinamento e teste, onde ambos possuem um conjunto de preditores (ou \textit{features}) e as variáveis que se desejam prever (ou classes). Este primeiro conjunto de dados irá alimentar o modelo para poder encontrar padrões nos dados, ao passo que o conjunto de teste servirá para validar se o modelo foi capaz de generalizar para dados não vistos no treinamento. \cite{zhou2021machine}

No entanto, existem alguns problemas com os quais essas abordagens precisam lidar, dois dos mais importantes são o viés e o \textit{overfitting}: o primeiro pode ocorrer quando as classes são muito desbalanceadas entre si, fazendo com que o modelo não seja capaz de encontrar padrões para diferenciar corretamente, o que pode implicar uma queda no desempenho. Já o \textit{overfitting} ocorre quando o modelo consegue aprender e prever com um bom desempenho, mas não é capaz de generalizar bem para novos dados. \cite{zhou2021machine}. A seguir, será apresentada a fundamentação teórica dos modelos de \textit{machine learning} utilizados neste trabalho.

\section{Modelagem Preditiva com Regressão}
\label{regression}

A análise preditiva utiliza técnicas de aprendizado de máquina para fazer previsões sobre eventos futuros com base em dados históricos. No domínio jurídico, isso pode se traduzir na previsão de desfechos de casos ou, como neste estudo, na estimativa de valores de indenização.

\subsection{Regressão}
A regressão é uma tarefa de aprendizado supervisionado cujo objetivo é prever um valor de saída contínuo e numérico. Diferentemente da classificação, que prevê uma categoria (e.g., "procedente" ou "improcedente"), a regressão estima uma quantidade, como o valor monetário de uma indenização por dano moral.

\subsection{Árvores de Decisão para Regressão}
Uma Árvore de Decisão (\textit{Decision Tree}) é um algoritmo de aprendizado de máquina que constrói um modelo preditivo na forma de uma estrutura de árvore. Cada nó interno da árvore representa uma pergunta sobre um dos atributos do caso (e.g., "O atraso do voo foi maior que 4 horas?"), e cada nó folha representa uma predição de valor. Uma das principais vantagens deste algoritmo é sua alta interpretabilidade, pois o fluxo de decisões que leva a uma previsão pode ser facilmente visualizado e compreendido. O modelo desenvolvido neste trabalho, o \textit{DecisionTreeRegressor}, é uma implementação deste conceito para tarefas de regressão.

\section{Inteligência Artificial Explicável (XAI)}
\label{xai}

A crescente utilização de modelos de IA em áreas críticas como o Direito levanta a questão da transparência. Muitos algoritmos complexos funcionam como uma "caixa-preta" (\textit{black box}), tornando difícil para um ser humano compreender como uma decisão específica foi alcançada. A Inteligência Artificial Explicável (XAI) é o campo de pesquisa que busca desenvolver técnicas para tornar os modelos de IA mais transparentes e interpretáveis.

\subsection{SHapley Additive exPlanations (SHAP)}
O \textit{SHAP} é um \textit{framework} unificado para a interpretação de modelos preditivos, baseado no conceito de valores de \textit{Shapley} da teoria dos jogos cooperativos \cite{salih2025perspective}. Ele quantifica a contribuição de cada fator (ou \textit{feature}) do caso para o resultado da predição. Para uma determinada previsão, o \textit{SHAP} calcula como cada informação (e.g., o tempo de atraso do voo, a ocorrência de extravio de bagagem) empurrou o valor final para cima ou para baixo em relação a um valor base (a média das predições). Isso permite gerar explicações tanto para casos individuais (explicações locais) quanto para o comportamento geral do modelo (explicações globais), sendo uma ferramenta poderosa para garantir a transparência e a confiabilidade das decisões algorítmicas.
\chapter{Trabalhos relacionados}
\label{ch:trabalhos_relacionados}

A aplicação de Inteligência Artificial no domínio jurídico, embora promissora, não é um campo inexplorado. Diversos pesquisadores, tanto no cenário nacional quanto internacional, têm se dedicado a desenvolver modelos preditivos para uma variedade de tarefas. Esta seção posiciona o presente trabalho em relação à literatura existente, destacando as sinergias e, sobretudo, as lacunas que esta pesquisa busca preencher, com foco na predição de valores de indenização por meio de regressão e na aplicação de técnicas de explicabilidade.

\section{Análise Preditiva de Decisões Judiciais}

A predição de resultados judiciais é uma das áreas mais investigadas na interseção entre IA e Direito. Um dos trabalhos pioneiros nessa frente, conduzido por \citeonline{aletras2016predicting}, utilizou técnicas de Processamento de Linguagem Natural (PLN) e modelos de \textit{Support Vector Machine} (SVM) para prever as decisões do Tribunal Europeu dos Direitos Humanos, alcançando uma acurácia de 79\% e demonstrando a viabilidade de usar o conteúdo textual das petições para antever seus desfechos.

No contexto brasileiro, a pesquisa de \citeonline{liu2017predictive} comparou o desempenho de diferentes classificadores, como Redes Neurais e Árvores de Regressão, para prever o resultado de casos de homicídio e corrupção, concluindo que a escolha do algoritmo ideal depende fortemente do subconjunto de dados analisado. Na mesma linha, \citeonline{magalhaes2022tecnicas} exploraram diversas técnicas de aprendizado de máquina para a tarefa de classificação de decisões judiciais, consolidando a validade desses métodos para a análise de dados jurídicos no Brasil. Esses trabalhos, embora fundamentais, concentram-se na tarefa de \textit{classificação} (prever um resultado categórico), enquanto a presente pesquisa se dedica à \textit{regressão} (prever um valor contínuo).

\section{Regressão Aplicada à Predição de Indenizações}

A predição de valores monetários em sentenças judiciais, uma tarefa de regressão, é mais específica e menos explorada. A pesquisa que mais se aproxima deste projeto foi desenvolvida por \citeonline{hsieh2021legal}, que testaram e compararam múltiplos algoritmos de regressão para prever o valor de indenização por "sofrimento mental" em casos de acidentes de carro fatais em Taiwan. Os autores concluíram que o modelo \textit{Random Forest} apresentou o melhor desempenho. Embora tenham realizado uma análise de importância das \textit{features}, uma técnica mais simples de explicabilidade, eles não aplicaram métodos de XAI avançados como o \textit{SHAP}.

No âmbito do próprio projeto Concil-IA, do qual este trabalho faz parte, pesquisas anteriores pavimentaram o caminho para a modelagem aqui proposta. O trabalho de \citeonline{sabo2022clustering} utilizou técnicas de clusterização para explorar a mesma base de sentenças sobre falhas no transporte aéreo, sendo fundamental para a identificação inicial dos fatores de maior relevância. Posteriormente, \citeonline{dal2023regression} desenvolveram um dos primeiros modelos de regressão para prever o dano moral nesse mesmo contexto, validando a abordagem e estabelecendo uma base de comparação de desempenho.

Mais recentemente, a frente de extração de fatores do projeto, documentada em \citeonline{pereira2025using}, explorou o uso de Grandes Modelos de Linguagem (\textit{LLMs}), como o \textit{GPT-4o}, para automatizar a extração de variáveis das sentenças judiciais, processo que gerou dados complementares aos utilizados para treinar o modelo aqui apresentado (embora estes não pertençam ao escopo abordado).

\section{A Lacuna da Explicabilidade e a Contribuição Deste Trabalho}

Apesar dos avanços na predição, a questão da transparência permanece um desafio central. A preocupação com o efeito "caixa-preta" dos modelos de IA no Direito é amplamente discutida na literatura. Embora trabalhos como o de \citeonline{hsieh2021legal} tenham abordado a importância dos fatores, a aplicação de métodos robustos de XAI, como o \textit{SHAP}, ainda é incipiente na área, especialmente em modelos de regressão para o Direito do Consumidor brasileiro.

Diante do exposto, a principal contribuição deste trabalho é preencher essa lacuna, ao integrar sistematicamente três áreas:
\begin{enumerate}
    \item \textbf{A modelagem de regressão} para predição de valores de dano moral, aprofundando os resultados obtidos por \citeonline{dal2023regression};
    \item \textbf{O contexto específico do Direito do Consumidor} brasileiro, utilizando uma base de dados real e validada;
    \item \textbf{A aplicação de uma técnica de XAI avançada (\textit{SHAP})} para fornecer explicações locais e globais para as predições do modelo, tornando-o transparente e auditável para o usuário final.
\end{enumerate}

Dessa forma, esta pesquisa não apenas desenvolve um modelo preditivo, mas o apresenta como uma ferramenta explicável, alinhada às diretrizes éticas para o uso de IA no Judiciário, e com potencial de aplicação em plataformas de Resolução de Disputas Online (ODR) para auxiliar no processo de conciliação.
\chapter{Metodologia}
\label{ch:metodologia}
O desenvolvimento deste trabalho seguiu uma abordagem quantitativa para o modelo e qualitativa para a explicabilidade;
com a aplicação de técnicas de Ciência de Dados e Aprendizado de Máquina
para a construção de um modelo preditivo de regressão e técnicas de Inteligência Artificial Explicável
para permitir a explicação e interpretação de dados por usuários leigos.

O processo metodológico do Projeto Concil-IA como um todo\ foi estruturado em um \textit{pipeline} de múltiplas etapas como observa-se na figura \ref{fig:concilia_pipeline}

\begin{figure}[ht]
    \centering
    \includegraphics[width=0.5\linewidth]{images/Concil-IA-Pipeline.png}
    \caption{Pipeline Concil-IA}
    \fonte{Elaborado pelo Autor}
    \label{fig:concilia_pipeline}
\end{figure}

No escopo específico deste trabalho, o foco recai sobre as etapas de processamento, modelagem e explicabilidade, conforme destacado no fluxo da Figura \ref{fig:project-pipeline}
Já este trabalho contempla apenas as etapas:
\begin{figure}[ht]
    \centering
    \includegraphics[width=0.5\linewidth]{images/Project-Pipeline.png}
    \caption{Pipeline desse projeto}
    \fonte{Elaborado pelo Autor}
    \label{fig:project-pipeline}
\end{figure}

As seções subsequentes detalham os recursos utilizados e as etapas percorridas, partindo da origem dos dados jurídicos (\ref{sec:base_dados}), passando pelas ferramentas computacionais empregadas, e culminando no fluxo de engenharia de atributos e treinamento dos modelos inteligentes

\section{Base de Dados e Ferramentas Computacionais}
\label{sec:base_dados}
A fundamentação empírica deste estudo reside em um conjunto de sentenças judiciais reais e no ecossistema de bibliotecas da linguagem Python, escolhidos para garantir a reprodutibilidade e a robustez das análises.

A seguir, detalha-se a composição do \textit{dataset} e o ambiente de desenvolvimento.

\subsection{Base de Dados}
\label{sub:dataset}
A base de dados é composta por 1.174 sentenças proferidas pelo Juizado Especial Cível da UFSC, versando especificamente sobre falhas na prestação de serviços de transporte aéreo à luz do Código de Defesa do Consumidor.

Os dados, anonimizados em conformidade com a Lei Geral de Proteção de Dados (LGPD), foram estruturados e extraídos para arquivos no formato CSV (\textit{comma-separated values}).
Após a anonimização, que garantiu a supressão de identificadores sensíveis das partes, procedeu-se à extração das variáveis (ou \textit{features}) relevantes para a quantificação do dano.
A tabela \ref{tab:feature_desc} apresenta as variáveis estruturadas utilizadas neste estudo.

\begin{table}[ht]
    \centering
    \begin{tabular}{|| m{6 cm} | m{10 cm} ||}
        \hline
        Feature & Descrição \\
        \hline
        \hline
        direito de arrependimento & O consumidor tentou cancelar ou alterar a
        compra no prazo de 7 dias, mas não lhe foi permitido fazê-lo. \\ \hline
        descumprimento de oferta & A empresa fornecedora não cumpriu com o
        ofertado ao consumidor(a), seja no valor da passagem, seja no assento no
        avião ou situações correlatas. \\ \hline
        extravio definitivo & Uma ou mais bagagens foram extraviadas e nunca
        foram recuperadas. \\ \hline
        extravio temporário & Uma ou mais bagagens foram extraviadas e
        posteriormente recuperadas. \\ \hline
        intervalo extravio temporário & O tempo decorrido até recuperar a
        bagagem, caso haja extravio temporário \\ \hline
        violação, furto ou avaria & Adulteração da bagagem ou um item dela \\ \hline
        cancelamento/alteração de destino & O consumidor não foi levado ao
        destino inicial \\ \hline
        atraso de vôo & O consumidor foi levado ao destino inicial, mas com
        atraso maior que 4 horas \\ \hline
        intervalo atraso & O tempo decorrido até o consumidor chegar ao seu
        destino, caso haja atraso de vôo \\ \hline
        culpa exclusiva consumidor & Todos os problemas relatados decorrem de
        ações ou inações do consumidor. \\ \hline
        condições climáticas / fechamento & Evento imprevisível e inevitável que
        comprovadamente impediu o aeroporto de operar. \\ \hline
        no show & Cancelamento automático do vôo de retorno em razão exclusiva
        do não comparecimento à viagem de ida, sem concordância do
        consumidor. \\ \hline
        overbooking & Venda de assentos em quantidade superior a comportada pela
        aeronave fazendo com que o consumidor não embarcasse. \\ \hline
        assistência cia aérea & A companhia aérea forneceu auxílio concreto ao
        consumidor para enfrentar o problema de atraso de voo ou cancelamento
        sem realocação/alteração de destino, e essa assistência inclui
        hospedagem, alimentação, ou transporte alternativo. \\ \hline
        hipervulnerável & O consumidor era idoso, gestante, possuía deficiência,
        fazia uso de medicamentos ou estava acompanhado de criança (entende-se
        criança a(o) infante que tem até 12 anos incompletos). \\ \hline
    \end{tabular}
    \caption{Descrição das Features}
    \fonte{Elaborado pelo autor}
    \label{tab:feature_desc}
\end{table}

Ressalta-se que, embora o projeto Concil-IA possua iniciativas de extração automatizada de dados \cite{pereira2025using}, conforme descrito na tabela \ref{tab:extraction_modes}, este trabalho utiliza exclusivamente o conjunto de dados extraído e validado manualmente (\textit{Gold Standard}), visando minimizar ruídos decorrentes de erros de interpretação automática de texto nesta fase de validação do modelo.

\begin{table}[!ht]
    \centering
    \begin{tabular}{|| m{4 cm} | m{4 cm} ||}
        \hline
        Modo de Extração & Número de Entradas \\ \hline
        \hline
        \hline
        Manual & 1174 \\ \hline
        Automática & 687 \\ \hline
    \end{tabular}
    \caption{Proporção de modalidades de extração}
    \fonte{Elaborado pelo Autor}
    \label{tab:extraction_modes}
\end{table}

A tabela \ref{tab:features_distribution} mostra a distribuição de valores para cada feature encontrada na base de dados manualmente extraída.

\begin{table}[!ht]
    \centering
    \begin{tabular}{|| m{6 cm} | m{2 cm} | m{2 cm} ||}
        \hline
        Features: & N & S \\ \hline
        \hline
        \hline
        direito de arrependimento & 98,65\% & 1,35\% \\ \hline
        descumprimento de oferta & 99,15\% & 0,85\% \\ \hline
        extravio definitivo & 96,86\% & 3,14\% \\ \hline
        extravio temporário & 86,84\% & 13,16\% \\ \hline
        violação furto avaria & 95,67\% & 4,33\% \\ \hline
        cancelamento & 85,48\% & 14,52\% \\ \hline
        atraso & 51,19\% & 48,81\% \\ \hline
        culpa exclusiva do consumidor & 98,39\% & 1,61\% \\ \hline
        fechamento do aeroporto & 98,64\% & 1,36\% \\ \hline
        no show & 96,43\% & 3,56\% \\ \hline
        overbooking & 96,86\% & 3,14\% \\ \hline
        assistência cia aérea & 83,53\% & 16,47\% \\ \hline
        hipervulnerável & 96,10\% & 3,90\% \\ \hline
    \end{tabular}
    \caption{Distribuição dos features na base de dados manualmente extraída}
    \fonte{Elaborado pelo autor}
    \label{tab:features_distribution}
\end{table}

\subsection{Ferramentas Computacionais}

A implementação do \textit{pipeline} de Ciência de Dados foi realizada na linguagem Python (versão 3.13), selecionada por sua ampla comunidade e suporte a bibliotecas de Aprendizado de Máquina. A tabela \ref{tab:python_libs} resume as principais bibliotecas empregadas.

\begin{table}[!ht]
    \centering
    \begin{tabular}{|| m{3 cm} | m{5 cm}| m{5 cm} ||}
        \hline
        Categoria & Ferramentas/Bibliotecas & Usabilidade Principal \\
        \hline
        \hline
        Manipulação e tratamento de dados & pandas, numpy,
        imblearn.over\_sampling .RandomOverSampler & Limpeza, transformação,
        balanceamento de dados \\ \hline
        Modelagem preditiva & scikit-learn (DecisionTreeRegressor,
        RandomForestRegressor, AdaBoostRegressor, train\_test\_split) &
        Construção, avaliação e seleção de modelos de regressão \\ \hline
        Visualização e exportação & matplotlib, graphviz, tree.plot\_tree,
        tree.export\_graphviz, joblib & Geração de gráficos e visualização de
        árvores de decisão \\ \hline
        Avaliação de desempenho & scikit-learn (metrics.classification\_report),
        RMSE, MAE & Medição de desempenho dos modelos \\ \hline
        Explicabilidade (XAI) & shap (shap.Explainer, shap.plots.waterfall,
        shap.plots.bar) & Interpretação e explicação dos resultados dos
        modelos \\ \hline
        Utilitários e suporte & joblib, json, tqdm, packaging, python-dateutil,
        pytz, typing\_extensions, tzdata & Suporte à execução e ao ambiente
        computacional \\ \hline
    \end{tabular}
    \caption{Bibliotecas python utilizadas}
    \fonte{Elaborado pelo autor}
    \label{tab:python_libs}
\end{table}

As versões, uso específico, detalhamento e bibliotecas adicionais encontram-se no Apêndice \ref{tab:python_versions}

\section{Pipeline de Desenvolvimento do Modelo}

O desenvolvimento do modelo preditivo seguiu um fluxo iterativo composto por três grandes fases
\begin{enumerate}
    \item O pré-processamento dos dados brutos,
    \item O treinamento e seleção de algoritmos,
    \item A implementação da camada de explicabilidade.
\end{enumerate}

Cada uma destas fases envolveu decisões de projeto visando equilibrar a precisão estatística com a coerência jurídica.

\subsection{Pré-processamento e Engenharia de Atributos}

Como detalhado na seção \ref{sub:dataset}, a base de dados pela qual o modelo se guiará foi extraída de sentenças jurídicas (anonimizadas) de processo a companhias aéreas para tabelas de dados estruturados.

Inicialmente, realizou-se a discretização das variáveis contínuas de tempo (atraso e extravio de bagagem), conforme os intervalos tipicamente utilizados na jurisprudência, apresentados nas tabelas \ref{tab:faixas-intervalo-de-extravio} e \ref{tab:faixas-intervalo-de-atraso}.

\begin{table}
    \centering
    \begin{tabular}{|| m{1.5 cm} | m{3 cm} | m{4.5 cm} ||}
        \hline
        Faixa &
        Intervalo do Extravio (Horas) &
        Frequência na Base de Dados
        Manualmente Extraída \\
        \hline
        \hline
        0 & 0             & 86,59 \% \\ \hline
        1 & 1 - 24      & 3,06 \% \\ \hline
        2 & 25 - 72   & 4,57 \% \\ \hline
        3 & 73 - 168 & 3,23 \% \\ \hline
        4 & 169         & 2,55 \% \\ \hline
    \end{tabular}
    \caption{Faixas de Intervalo de Extravio}
    \fonte{Elaborado pelo autor}
    \label{tab:faixas-intervalo-de-extravio}
\end{table}
\begin{table}[!ht]
    \centering
    \begin{tabular}{|| | m{4 cm} | m{4 cm} | m{4 cm} ||}
        \hline
        Faixa & Intervalo do Atraso (HH:MM) & Frequência na Base de Dados Manualmente Extraída \\
        \hline
        \hline
        -1 & Consumidor não chegou ao destino & 14,52 \% \\ \hline
        0 & 0 & 50,25 \% \\ \hline
        1 & 0:01 - 3:59 & 2,38 \% \\ \hline
        2 & 4:00 - 7:59:00 & 10,70 \% \\ \hline
        3 & 8:00 - 11:59 & 10,78 \% \\ \hline
        4 & 12:00 - 15:59 & 7,56 \% \\ \hline
        5 & 16:00 - 23:59 & 6,03 \% \\ \hline
        6 & 24:00 - 27:59 & 8,74 \% \\ \hline
        7 & 28:00 & 3,56 \% \\ \hline
    \end{tabular}
    \caption{Faixas utilizadas no feature intervalo de atraso}
    \fonte{Elaborado pelo autor}
    \label{tab:faixas-intervalo-de-atraso}
\end{table}


\paragraph{Tratamento de Variáveis e Filtros Lógicos}
Realizou-se a conversão de variáveis categóricas para formatos numéricos categóricos e a remoção estratégica de colunas.

\textit{Features} categóricos foram convertidos em numéricos mediante substituição binária 
--- "Sim" ou "S" é substituído por '1', 'Não' ou "N" por '0', valores não identificados por '0' também ---,
ao passo que atributos contínuos, como os intervalos temporais mencionados acima, foram discretizados em faixas baseadas na distribuição observada, usando o método dos quartis (\cite{pinheiro2009estatistica}).

Optou-se pela exclusão de casos improcedentes e de variáveis que atuam como "fatores de anulação" do dano moral.
Essa decisão justifica-se pois tais fatores são determinísticos: sua presença implica, juridicamente, a inexistência de dever de indenizar.

Portanto, os \textit{features} "culpa exclusiva do consumidor" e "condições climáticas/fechamento do aeroporto" são verificadas na interface do \textit{website} e não no modelo. --- Ou seja, antes do usuário poder inserir os fatores específicos do seu caso, ele é questionado sobre as condições climáticas do momento de seu vôo e se os problemas não decorrem de suas ações.
Caso responda sim, é informado diretamente que a indenização por dano moral provavelmente será nula e desencorajado de usar o modelo.

% [TODO 2: Feature Selection virá aqui]
% \cite{kuhn2013applied}
Adicionalmente, aplicou-se uma inversão lógica no atributo "assistência da companhia aérea". Originalmente um fator atenuante, ele foi transformado na variável "desamparo", que é contabilizada apenas quando a empresa falha em prestar auxílio.
Com isso, buscou-se uma monotonicidade positiva, onde a presença de qualquer \textit{feature} no vetor de entrada contribui positivamente para o aumento do valor predito.

Os \textit{features} "extravio temporário de bagagem" (binário) e "intervalo de extravio de bagagem" foram combinados (se não houve extravio de bagagem, o intervalo é 0). Da mesma forma, "atraso" e "intervalo de atraso" foram combinados, e depois o \textit{feature} "cancelamento/alteração de destino" também foi combinada em "intervalo de atraso", tornando-se a faixa '-1', para representar infinito (pois o consumidor nunca chegou ao destino contratado).

Variáveis redundantes ou correlatas foram fundidas ou removidas para reduzir a dimensionalidade.
Os \textit{features} binários "extravio temporário de bagagem" e "atraso" foram removidos, uma vez que sua informação já é contida nas variáveis de intervalo.

Já o \textit{Feature} "cancelamento de voo" foi incorporado à variável de tempo como um valor de atraso infinito (faixa '-1'), simplificando a interpretação do modelo.

\paragraph{Tratamento de Outliers e Balanceamento}
Valores discrepantes (\textit{outliers} \cite{hawkins1980identification}) foram identificados pelo método dos quantis \cite{wilcox2012introduction} e tratados com remoção seletiva.

Para mitigar o efeito de desbalanceamentos de classes, aplicou-se a técnica \textit{RandomOverSampler} (da biblioteca \textit{Python} \textit{imblearn}), com a estratégia "\textit{not-majority}" para uniformizar as categorias-alvo (14, decididas baseado na distribuição de valores alvo vistas na Figura \ref{fig:value_distribution}).

\begin{figure}[h]
    \centering
    \includegraphics[width=0.5\linewidth]{images/dano-moral-distribuicao.png}
    \caption{Distribuição de Valores para o Dano moral}
    \label{fig:value_distribution}
    \fonte{Elaborado pelo Autor}
\end{figure}

% [TODO 2: Corrigir o balanceamento no código e atualizar aqui]
Reconhece-se que aplicar o balanceamento antes de separar os conjuntos de treino e teste é uma falha metodológica, uma vez que introduz vieses se o conjunto de teste incluir dados replicados, e será corrigida futuramente.

\subsection{Treinamento e Seleção do Modelo}

O conjunto de dados foi dividido em 80\% para treinamento e 20\% para teste, utilizando uma estratégia de amostragem estratificada para garantir que a distribuição das classes de valor fosse preservada em ambas as amostras. 

% [TODO 2: Depois de mudar a metodologia para incluir um conjunto de validação, citar o Machine Yearning]
Nesta primeira etapa, foram treinados e avaliados três algoritmos de regressão da biblioteca \textit{scikit-learn}:
\textit{DecisionTreeRegressor} (\cite{blockeel2023decision}), 
\textit{RandomForestRegressor} (\cite{zhang2023compare}) 
e \textit{AdaBoostRegressor} (\cite{airlangga2024anomaly})

A otimização dos hiperparâmetros de cada modelo foi realizada por meio de uma abordagem heurística, com a variação de um hiperparâmetro por vez, devido às limitações de recursos computacionais disponíveis.

\subsection{Avaliação de Desempenho}
\label{sub:desempenho}
A validação do modelo não se restringiu à precisão numérica, incorporando também critérios de transparência algorítmica e performance computacional

Utilizou-se o Erro Quadrático Médio Raiz (RMSE) para penalizar grandes desvios
e o Erro Absoluto Médio (MAE) para mensurar a margem de erro média em valores monetários reais \cite{zhang2023compare}.

Além do desempenho preditivo, a seleção do modelo final considerou fatores secundários,
como o custo computacional (memória e tempo de processamento
e a interpretabilidade inerente de cada algoritmo,
sendo este último um critério fundamental para os objetivos do projeto.

Nesse contexto, com os 3 modelos possuindo desempenho similar, os fatores secundários foram determinantes na escolha final.

\subsection{Implementação da Explicabilidade (XAI)}
Para garantir a transparência das predições, foi implementado o \textit{framework} SHAP (\textit{SHapley Additive exPlanations} (\cite{lundberg2017unified})).

Utilizando o método \textit{Explainer}, o sistema calcula a contribuição marginal de cada fato jurídico (ex: tempo de atraso, extravio) para o valor final da indenização, gerando visualizações como o \textit{waterfall plot}.

Isso permite que o usuário compreenda não apenas "quanto" receberá, mas "por que" aquele valor foi sugerido.
\chapter{Resultados Preliminares e discussão}

Este capítulo apresenta os resultados obtidos a partir da aplicação da metodologia descrita anteriormente.
A exposição está dividida em duas seções principais. A primeira seção aborda o desempenho quantitativo dos modelos de regressão treinados, detalhando as métricas de avaliação e justificando a escolha do modelo final. A segunda seção dedica-se à análise qualitativa da explicabilidade do modelo selecionado, demonstrando como as técnicas de XAI permitem interpretar as predições geradas.

\section{Desempenho dos Modelos de Regressão}

Conforme a metodologia, três algoritmos de regressão foram treinados e avaliados utilizando o conjunto de dados de 1.174 sentenças extraídas manualmente. O desempenho de cada modelo foi mensurado pelas métricas de Erro Quadrático Médio Raiz (RMSE) e Erro Absoluto Médio (MAE). Os resultados consolidados estão apresentados na Tabela \ref{tab:metricas}.

% [TODO 2: Obter os resultados para isso e outros modelos pertinentes]
\begin{table}[!ht]
    \centering
    \begin{tabular}{|| m{7 cm} | m{2 cm} | m{2 cm} ||}
        \hline
        Modelo & RMSE  & MAE\\ \hline
        \hline
        \hline
        Decision Tree Regressor (Dados Manualmente Extraídos) & 2260.57 &
        1577.21 \\ \hline
        Decision Tree Regressor (Todos os Dados Extraídos) & 2285.71 &
        1672.10 \\ \hline
        Random Forest Regressor (Dados Manualmente Extraídos) & 2262.12 &
        1586.24 \\ \hline
    \end{tabular}
    \caption{Resultados alcançados com os modelos testados}
    \fonte{Elaborado pelo autor}
    \label{tab:metricas}
\end{table}

A análise comparativa dos resultados indica um desempenho muito similar entre os modelos \textit{Decision Tree Regressor} e \textit{Random Forest Regressor}, com uma ligeira vantagem para o primeiro em ambas as métricas.
O modelo \textit{AdaBoost Regressor}, por sua vez, foi descartado em uma fase inicial de testes, pois, embora apresentasse um desempenho de erro próximo ao dos outros modelos, seu consumo de memória era aproximadamente 300 vezes superior ao do \textit{Decision Tree}, tornando-o inviável dadas as restrições de recursos computacionais do projeto, \ref{sub:desempenho}.

% [TODO 2: Mostrar relatórios de performance para poder dizer que o peso dos modelos contou]
Diante do exposto, o modelo \textbf{Decision Tree Regressor} foi selecionado para as etapas subsequentes de explicabilidade.

A escolha foi pautada por três fatores principais:
\begin{enumerate}
    \item \textbf{Melhor Desempenho:} Apresentou os menores valores de RMSE e MAE comparado ao \textit{Random Forest Regressor}, ainda que por uma margem estreita.
    \item \textbf{Eficiência Computacional:} Exigiu significativamente menos recursos de memória e processamento em comparação com os modelos de \textit{ensemble} (\textit{Random Forest} e \textit{AdaBoost}).
    \item \textbf{Interpretabilidade Inerente:} Como um modelo de árvore única, sua estrutura de funcionamento pode ser visualizada diretamente, o que se alinha ao objetivo central do projeto de garantir a máxima transparência e interpretabilidade.
\end{enumerate}

\section{Análise da Explicabilidade do Modelo (XAI)}

Após a seleção do \textit{Decision Tree Regressor}, foi aplicado o \textit{framework} SHAP para analisar como o modelo utiliza os fatores do caso para realizar suas predições. Esta análise foi conduzida em duas frentes: a importância global dos atributos e a explicação de um caso específico.

\subsection{Importância Global dos Atributos}

A análise global permite compreender quais fatores, em média, possuem maior impacto nas predições do modelo em todo o conjunto de dados. O gráfico de barras apresentado em \ref{fig:global_feature_importance} ilustra a importância média de cada variável.

\begin{figure}[h]
    \centering
    \includegraphics[width=0.5\linewidth]{images/global_feature_importance.png}
    \caption{Global Feature Importance}
    \label{fig:global_feature_importance}
    \fonte{Elaborado pelo Autor}
\end{figure}

Observa-se que o fator mais influente para a determinação do valor do dano moral é o \textbf{Intervalo de Atraso}, que representa o tempo de atraso do voo ou o seu cancelamento.
Em segundo lugar, destaca-se o \textbf{Desamparo}, variável que indica se a companhia aérea prestou ou não a devida assistência material ao consumidor.
Esses dois fatores são, de longe, os mais decisivos para o modelo.
A alta frequência dessas variáveis na base de dados, conforme \ref{fig:value_distribution}, contribui para que o modelo aprenda com maior robustez sobre seus impactos.

O gráfico de enxame (bee-swarm plot) de \ref{fig:global_bee_swarm} complementa essa análise, mostrando não apenas a importância, mas também a direção do impacto de cada fator. Nele, é possível visualizar que valores altos de "Intervalo de Atraso" e a presença de "Desamparo" (valores em vermelho) tendem a empurrar a predição para valores de indenização mais altos.

% [TODO 2: Pesquisar mais como interpretar o gráfico SHAP]

\begin{figure}[h]
    \centering
    \includegraphics[width=0.5\linewidth]{images/BeeSwarm.png}
    \caption{Global Feature Importance}
    \label{fig:global_bee_swarm}
    \fonte{Elaborado pelo Autor}
\end{figure}

\subsection{Análise de um Caso Específico (Explicação Local)}

% [TODO 2: Refazer isso com um novo modelo]

A principal vantagem do SHAP é sua capacidade de gerar explicações para predições individuais.
\ref{fig:shap_example_case} apresenta um gráfico de cascata (\textit{waterfall plot}) que detalha como o modelo chegou à predição de R\$ 9.290,44 para um caso específico da base de dados.

A interpretação do gráfico se dá da seguinte forma:
\begin{itemize}
    \item O ponto de partida é o valor base (\textit{E[f(x)] = 7.934.49}), que representa a média de todas as predições do modelo.
    \item Cada fator do caso em análise adiciona (em vermelho) ou subtrai (em azul) um valor a essa base.
    \item No exemplo, o fato de o \textbf{Intervalo de Atraso} ter sido de 15 a 24 horas contribuiu com um acréscimo de R\$ 929,59 ao valor da indenização.
    \item O \textbf{Desamparo} por parte da companhia aérea adicionou mais R\$ 391,15.
    \item Os demais fatores, por não estarem presentes neste caso específico, tiveram impacto nulo.
    \item A soma de todas essas contribuições resulta no valor final predito pelo modelo (\textit{f(x)} = R\$ 9.290,44).
\end{itemize}

Essa análise local demonstra a capacidade da ferramenta de fornecer uma justificativa transparente e quantificada para cada estimativa, tornando o resultado do algoritmo compreensível para um usuário leigo e adicionando transparência até mesmo para as decisões judiciais.

\begin{figure}[h]
    \centering
    \includegraphics[width=0.5\linewidth]{images/Waterfall.png}
    \caption{Shap Example}
    \label{fig:shap_example_case}
    \fonte{Elaborado pelo Autor}
\end{figure}

\section{Discussão dos Resultados Preliminares}

Os resultados quantitativos demonstram que foi possível desenvolver um modelo de regressão com uma margem de erro absoluta média de R\$ 1.577,21. Embora exista uma margem de erro inerente, o valor é considerado satisfatório para o propósito de uma ferramenta de auxílio à conciliação, onde se busca uma estimativa razoável, e não um valor exato.

Mais importante que a precisão numérica, contudo, é a capacidade do modelo de justificar suas predições de forma alinhada à intuição jurídica.
A análise de explicabilidade revelou que os fatores de maior impacto no modelo (atraso/cancelamento e falta de assistência) são, de fato, os elementos centrais na fundamentação das sentenças judiciais reais para casos de transporte aéreo.
Isso confere validade e confiabilidade ao modelo, não apenas como uma "caixa-preta" preditiva, mas como um sistema cujo raciocínio é transparente e auditável.
A combinação do desempenho preditivo com a interpretabilidade via SHAP resulta em uma ferramenta com potencial para auxiliar as partes em uma negociação, fornecendo estimativas seguras e transparentes de uma eventual indenização por dano moral.
\chapter{Considerações Preliminares}
\label{ch:conclusao}

Este trabalho apresentou o desenvolvimento e a validação preliminar de uma ferramenta de Inteligência Artificial Explicável voltada ao domínio jurídico, especificamente para o projeto Concil-IA.
O estudo foi motivado pela necessidade de reduzir a assimetria de informações em audiências de conciliação envolvendo transporte aéreo, propondo o uso de aprendizado de máquina para fornecer estimativas objetivas e transparentes de dano moral.

Os experimentos realizados até o momento permitem corroborar a hipótese central da pesquisa (em ambiente controlado): é viável utilizar modelos de regressão treinados em sentenças passadas para antecipar desfechos judiciais com margem de erro controlada, desde que aliados a técnicas de explicabilidade que garantam a transparência do processo decisório.

\section{Síntese dos Resultados Parciais}

Embora os resultados parciais insuficientes para adoção em larga escala, cada objetivo estabelecido (\ref{sub:objetivos_spec}) produziu ao menos uma prova de conceito capaz de fomentar futuras evoluções.

As etapas de treinamento (\textit{i} e \textit{ii}) resultaram na escolha do \textit{DecisionTreeRegressor} como a abordagem mais equilibrada.

Embora modelos mais complexos (dentre os testados) pudessem oferecer métricas marginalmente superiores, a Árvore de Decisão entregou um desempenho satisfatório (RMSE de R\$ 2.260,57 e MAE de R\$ 1.577,21) mantendo a simplicidade estrutural necessária para auditoria e interpretação.

A análise da importância das \textit{features}, realizada por meio do \textit{framework} SHAP, permitiu identificar os fatores de maior impacto nas decisões do modelo — como o tempo de atraso do voo e a falta de assistência da companhia aérea —, cumprindo o \textbf{Objetivo Específico (vi)}.

Todo o fluxo de processamento de dados e experimentação foi parcialmente documentado (objetivo específico \textit{iv}), criando uma base sólida para os próximos passos do projeto.

O modelo resultante, por ser funcional e explicável, está tecnicamente apto e foi integrado a uma plataforma web, o website do projeto Concil-IA, o que viabiliza o cumprimento do \textbf{Objetivo Específico (iii)} e disponibiliza a funcionalidade ao público.

Isso valida a ferramenta não apenas como um artefato de software, mas como um instrumento alinhado à lógica jurídica do Juizado Especial Cível.

\section{Limitações do Estudo Atual}

Apesar da validação da viabilidade técnica, o projeto encontrou limitações que restringiram o desempenho máximo alcançável nesta fase:

\begin{itemize}
    \item \textbf{Recursos Computacionais:} A escassez de poder de processamento impediu uma varredura exaustiva de hiperparâmetros (\textit{Grid Search} mais amplo) e modelos, o que sugere que mesmo o modelo atual ainda possui margem para otimização e que modelos mais adequados à tarefa podem ser explorados.
    \item \textbf{Volume de Dados:} A base de 1.174 sentenças, embora suficiente para provar o conceito, é considerada pequena para capturar a totalidade das nuances de casos atípicos (outliers). Isso restringe a capacidade de generalização do modelo em cenários menos frequentes e impede a aplicação de arquiteturas mais complexas e potencialmente mais precisas..
\end{itemize}
\section{Próximos Passos}
\label{sec:proximos_passos}

Considerando que a integração preliminar do modelo à interface web já foi realizada e que uma versão base de todos os objetivos já foi atingida, a próxima fase concentrar-se-á no aprimoramento técnico, estatístico e metodológico do "motor" de inteligência artificial.
O objetivo é garantir que o modelo final não seja apenas funcional, mas robusto, reprodutível e estatisticamente validado.

As atividades planejadas dividem-se em quatro eixos principais:

\subsection{Engenharia de Software e Reprodutibilidade}
Para viabilizar a experimentação em larga escala necessária no TCC II, a infraestrutura do projeto será profissionalizada:
\begin{itemize}
    \item \textbf{Automação de Experimentos:} Reestruturação do repositório para permitir a execução parametrizada de múltiplos testes simultâneos, facilitando a comparação sistemática entre diferentes configurações.
    \item \textbf{Rastreabilidade:} Desenvolvimento de um gerador automático de \textit{logs} e relatórios de performance (computacional e preditiva) para cada rodada de treinamento.
    \item \textbf{Documentação Técnica:} Criação de uma \textit{Wiki} detalhada do repositório, documentando o fluxo de dados e as decisões de implementação para garantir a reprodutibilidade científica.
\end{itemize}

\subsection{Refinamento de Dados e Pré-processamento}
A qualidade dos dados de entrada será refinada para lidar com a complexidade temporal e contextual dos processos:
\begin{itemize}
    \item \textbf{Correção Monetária:} Implementação de correção monetária dos valores de indenização para uma data base (e.g., 2025) utilizando índices oficiais como o IPCA, mitigando distorções causadas pela inflação em sentenças antigas.
    \item \textbf{Tratamento de Contextos Excepcionais:} Criação de \textit{features} específicas para sinalizar períodos anômalos (como a pandemia de COVID-19) ou perfis de julgamento específicos, visando isolar vieses temporais ou de varas judiciais.
    \item \textbf{Gestão de \textit{Outliers}:} Estudo aprofundado sobre a remoção ou segregação de \textit{outliers}. Planeja-se analisar os casos removidos separadamente e, potencialmente, criar modelos específicos para tratar esses casos extremos, evitando que distorçam o aprendizado do padrão geral.
    \item \textbf{Seleção de Atributos:} Aplicação de bibliotecas de \textit{Feature Selection} e \textit{Elimination} para reduzir a dimensionalidade e o ruído dos dados.
\end{itemize}

\subsection{Estratégias de Modelagem}
Será abandonada a abordagem puramente heurística em favor de uma busca sistemática pelo melhor algoritmo:
\begin{itemize}
    \item \textbf{Diversificação de Algoritmos:} Testes comparativos com uma gama mais ampla de modelos, incluindo Regressão Linear, \textit{Support Vector Machines} (SVM), \textit{Naive Bayes}, Redes Bayesianas, \textit{XGBoost} e Redes Neurais.
    \item \textbf{Modelos Particionados:} Investigação de arquiteturas de "divisão e conquista", onde modelos distintos são treinados para tipos específicos de lide (ex: um modelo para atraso/cancelamento e um modelo simplificado de médias para casos de menor complexidade intersecção entre variáveis) e também realizar um segundo treinamento combinando as partes em um modelo abrangente.
    \item \textbf{Otimização de Hiperparâmetros:} Execução de \textit{Grid Search} para cada um dos algoritmos testados, visando encontrar a configuração ótima de cada modelo.
    \item \textbf{Ponderação de Classes:} Testes com técnicas de \textit{Weight Classification} (pesos para instâncias ou penalização de erros) como alternativa às técnicas tradicionais de balanceamento de dados.
\end{itemize}

\subsection{Protocolos de Avaliação e Validação Estatística}
Por fim, o rigor na avaliação será intensificado para evitar vieses estatísticos:
\begin{itemize}
    \item \textbf{Divisão Estratificada de Dados:} Adoção de uma partição rígida de dados em Treino (60\%), Validação/Dev (20\%) e Teste (20\%). O conjunto de teste será isolado ("coce" ou \textit{holdout}) e utilizado apenas para a avaliação final, garantindo que não haja ajuste de hiperparâmetros sobre ele.
    \item \textbf{Métricas de Avaliação Personalizadas:} Estudo de métricas alternativas ao erro médio padrão, como variações do MAPE baseadas no valor predito ao invés do correto (o que permitiria melhor informar ao usuário a margem de erro do modelo).
    \item \textbf{Curvas de Aprendizado:} Plotagem de curvas de aprendizado para diagnosticar problemas de \textit{overfitting} ou \textit{underfitting}.
    \item \textbf{Validação Estatística do SHAP:} Implementação de testes estatísticos para verificar se as \textit{features} apontadas como importantes pelo SHAP possuem distribuições significativamente diferentes entre os casos de acerto e erro do modelo, validando a explicabilidade matemática frente à realidade dos dados. Além disso, mais pesquisa sobre o funcionamento do SHAP é necessária para interpretá-lo para o usuário leigo.
\end{itemize}

%%%%%%%%%%%%%%%%%%%%%%%%%%%%%
%%% Elementos pós-textuais                                        %%%
%%%%%%%%%%%%%%%%%%%%%%%%%%%%%

\postextual
\bibliography{bibliografia}

\apendices
\chapter{Instruções para execução do fluxo de trabalho}
\label{ch:apendice}

A ferramenta desenvolvida ao longo do presente projeto pode ser encontrada no link \href{https://concilia.ufsc.br/}{https://concilia.ufsc.br/}

% [TODO 2: Descrever a estrutura e diretórios do repositório]
O código para a reprodução do fluxo proposto pode ser encontrado no link $\href{https://github.com/ConcilIA-EGOV/predictor_model_in_law_judgments}{https://github.com/ConcilIA-EGOV/predictor_model_in_law_judgments}$.

% [TODO 2: Descrever a melhor como usar o(s) repositório(s)]
Para a correta execução, é necessário ter a versão 3.10 do \textit{Python}, E o gerenciador de dependências \textit{pip}.
Após instalado, crie um ambiente virtual e instale as dependências necessárias:


\begin{lstlisting}
  -  python -m venv caminho/para/novo/ambiente/virtual
  -  source caminho/para/novo/ambiente/virtual/bin/activate
  -  pip install -r requirements.txt
\end{lstlisting}

Com o ambiente virtual aberto e devidamente configurado, basta digitar o comando \textit{make} para ser informado das opções de uso.

------------------------------------------------------------------------------


% [TODO 2: Descrever a estrutura e uso do site]

% [? TODO: Entender melhor os apêndices]

\begin{table}[!ht]
    \centering
    \begin{tabular}{|| m{3 cm} | m{2 cm} | m{9.5 cm} ||}
        \hline
        Biblioteca & Versão  & Usabilidade Principal \\ \hline
        \hline
        \hline
        cloudpickle & 3.1.1 & Serialização de objetos Python \\ \hline
        contourpy & 1.3.2 & Geração de contornos para visualização \\ \hline
        cycler & 0.12.1 & Criação de ciclos para estilos de plotagem \\ \hline
        fonttools & 4.57.0 & Manipulação de fontes tipográficas \\ \hline
        graphviz & 0.20.3 & Visualização de grafos e árvores \\ \hline
        imbalanced-learn & 0.13.0 & Técnicas de balanceamento de dados \\ \hline
        imblearn & 0.0 & Interface para imbalanced-learn \\ \hline
        joblib & 1.5.0 & Persistência e paralelização de objetos Python \\ \hline
        kiwisolver & 1.4.8 & Resolução de restrições para gráficos \\ \hline
        llvmlite & 0.44.0 & Suporte à compilação JIT (usado pelo numba) \\ \hline
        matplotlib & 3.10.1 & Visualização de dados \\ \hline
        numba & 0.61.2 & Compilação JIT para aceleração de código Python \\ \hline
        numpy & 2.2.5 & Operações numéricas e manipulação de arrays \\ \hline
        packaging & 25.0 & Utilitários para empacotamento de pacotes Python \\ \hline
        pandas & 2.2.3 & Manipulação e análise de dados tabulares \\ \hline
        pillow & 11.2.1 & Processamento de imagens \\ \hline
        pip & 25.1.1 & Gerenciador de pacotes Python \\ \hline
        pyparsing & 3.2.3 & Análise sintática de expressões \\ \hline
        python-dateutil & 2.9.0 & Manipulação avançada de datas \\ \hline
        pytz & 2025.2 & Suporte a fusos horários \\ \hline
        scikit-learn & 1.6.1 & Modelagem preditiva e algoritmos de machine
        learning \\ \hline
        scipy & 1.15.2 & Computação científica e estatística \\ \hline
        shap & 0.47.2 & Explicabilidade de modelos de machine learning \\ \hline
        six & 1.17.0 & Compatibilidade entre Python 2 e 3 \\ \hline
        sklearn-compat & 0.1.3 & Compatibilidade entre versões do
        scikit-learn \\ \hline
        slicer & 0.0.8 & Utilitário para slicing de dados \\ \hline
        threadpoolctl & 3.6.0 & Controle de pools de threads \\ \hline
        tqdm & 4.67.1 & Barra de progresso para loops \\ \hline
        typing\_extensions & 4.13.2 & Extensões de tipagem para Python \\ \hline
        tzdata & 2025.2 & Base de dados de fusos horários \\ \hline
    \end{tabular}
    \caption{Detalhamento das bibliotecas python usadas}
    \fonte{Elaborado pelo autor}
    \label{tab:python_versions}
\end{table}


\end{document}

% \chapter{Exemplo de Apêndice}
% \label{ch:apendice}

% Uso de \cite em apêndice/anexo como se fossem antes do \bibliography{}.  NBR
% 14724 e NBR 6023, assim como os documentos da BU não especificam nada sobre
% citações dentro de apêndices/anexos. No entanto, em email trocado com a BU, a
% orientação foi de usar \cite{} normalmente e deixar que as referências sejam
% listadas na única bibliografia do documento, mesmo que esta esteja antes dos
% apêndices. A argumentação é que apêndices e anexos são numerados e fazem parte
% do documento, logo suas referências devem ser listadas como referências do
% documento. Além disso as normas não prevem segmentar as referências por
% capítulos.
% \cite{turing1937} \lipsum[1]




% Lista para ambiente algorithm
% \listofalgorithms*

% \begin{listadesimbolos}
%   $\gets$   & Atribuição \\
%   $\exists$   & Quantificação existencial \\
%   $\rightarrow$   & Implicação \\
%   $\wedge$   & E lógico \\
%   $\vee$   & Ou lógico \\
%   $\neg$   & Negação lógica \\
%   $\mapsto$   & Mapeia para \\
%   $\sqsubseteq$   & Subclasse (em ontologias) \\
%   $\subseteq$   & Subconjunto: $\forall x\;.\; x \in A \rightarrow x \in B$ \\
%   $\langle\ldots\rangle$ & Tupla \\
%   $\forall$   & Quantificação universal \\
%   mmmmm & Nenhum sentido, apenas estou aqui para demonstrar a largura máxima dessas colunas. Ao abrir o ambiente \texttt{listadesimbolos}, pode-se fornecer um argumento opcional indicando a largura da coluna da esquerda (o default é de 5em): \texttt{\textbackslash{}begin\{listadesimbolos\}[2cm] .... \textbackslash{}end\{listadesimbolos\}} \\
% \end{listadesimbolos}